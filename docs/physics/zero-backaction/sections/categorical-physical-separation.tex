\section{Categorical-Physical Separation}

\subsection{Partition Coordinate Derivation from Bounded Phase Space}

For a particle in a central potential $V(r)$, the phase space is six-dimensional: $(\mathbf{r}, \mathbf{p}) \in \mathbb{R}^3 \times \mathbb{R}^3$. Boundedness means:
\begin{equation}
E = \frac{p^2}{2m} + V(r) < 0 \quad \Rightarrow \quad r < r_{\max}, \, p < p_{\max}
\end{equation}

The bounded region can be partitioned into nested shells. The partition coordinates arise from this nesting:

\begin{enumerate}
\item \textbf{Principal coordinate $n$}: Depth of nesting (radial shells). Larger $n$ corresponds to larger $r$.
\item \textbf{Angular coordinate $\ell$}: Number of angular nodes. Larger $\ell$ corresponds to higher angular momentum.
\item \textbf{Magnetic coordinate $m$}: Orientation of angular momentum vector. Range: $m \in \{-\ell, \ldots, +\ell\}$.
\item \textbf{Spin coordinate $s$}: Intrinsic angular momentum (for fermions). Range: $s \in \{-1/2, +1/2\}$.
\end{enumerate}

These coordinates are discrete because partitions are discrete. The values are integers or half-integers determined by geometry.

\subsection{Physical Coordinates from Continuous Phase Space}

Physical coordinates $(x, y, z, p_x, p_y, p_z)$ are continuous real numbers describing the particle's location and momentum in phase space. They satisfy the canonical commutation relations:
\begin{equation}
[x_i, p_j] = i\hbar \delta_{ij}, \quad [x_i, x_j] = 0, \quad [p_i, p_j] = 0
\end{equation}

These coordinates have continuous spectra: any real value is allowed (within the bounded region).

\subsection{Orthogonal Basis Theorem}

\begin{theorem}[Partition-Position Orthogonality]
Partition coordinates $(n, \ell, m, s)$ and physical coordinates $(x, y, z)$ form orthogonal bases for the Hilbert space:
\begin{equation}
\mathcal{H} = \text{span}\{|n,\ell,m,s\rangle\} = \text{span}\{|\mathbf{r}\rangle\}
\end{equation}
and the two bases are related by:
\begin{equation}
|\mathbf{r}\rangle = \sum_{n,\ell,m,s} \psi_{n\ell ms}(\mathbf{r}) |n,\ell,m,s\rangle
\end{equation}
\end{theorem}

The wavefunctions $\psi_{n\ell ms}(\mathbf{r})$ are the hydrogenic orbitals (for hydrogen) or analogous functions (for other systems). They provide the bijection between categorical and physical representations.

\subsection{Resolution Hierarchy}

Categorical coordinates provide coarse-grained information:
\begin{equation}
n \quad \Rightarrow \quad r \in [r_{\min}(n), r_{\max}(n)] \approx [(n-1)^2 a_0, n^2 a_0]
\end{equation}

Physical coordinates provide fine-grained information:
\begin{equation}
x \quad \Rightarrow \quad x \in [x - \Delta x/2, x + \Delta x/2]
\end{equation}
where $\Delta x$ can be arbitrarily small (limited only by Heisenberg uncertainty).

The resolution hierarchy is:
\begin{equation}
\Delta x_{\text{cat}} \sim n^2 a_0 \gg \Delta x_{\text{phys}} \sim \Delta x_{\text{measurement}}
\end{equation}

Categorical measurement sacrifices fine resolution for zero backaction.

\subsection{Information Content}

The information content of categorical coordinates is:
\begin{equation}
I_{\text{cat}} = \log_3 N_{\text{partitions}} \quad \text{trits}
\end{equation}

For hydrogen with $n \leq n_{\max}$:
\begin{equation}
N_{\text{partitions}} = \sum_{n=1}^{n_{\max}} 2n^2 = \frac{2n_{\max}(n_{\max}+1)(2n_{\max}+1)}{6} \approx \frac{2n_{\max}^3}{3}
\end{equation}

So $I_{\text{cat}} \approx \log_3(2n_{\max}^3/3) = 3\log_3 n_{\max} + \log_3(2/3)$ trits.

The information content of physical coordinates is infinite (continuous spectrum).

\subsection{Measurement Domains}

\textbf{Categorical measurement answers}: "Which partition?"

Example: Measuring $n = 2$ means the electron is in the second radial shell, somewhere in $r \in [a_0, 4a_0]$.

\textbf{Physical measurement answers}: "Where exactly?"

Example: Measuring $x = 2.3 a_0 \pm 0.1 a_0$ means the electron is precisely at that location.

The two questions are independent. Knowing the partition (categorical) does not determine the exact position (physical), and vice versa (though measuring position does determine which partition, the mapping is many-to-one).

\subsection{Bijection But Not Identity}

There is a bijection $f: \text{Partitions} \to \text{Regions}$ given by:
\begin{equation}
(n, \ell, m, s) \mapsto \Omega_{n\ell ms} = \{\mathbf{r} : |\psi_{n\ell ms}(\mathbf{r})|^2 > \text{threshold}\}
\end{equation}

But the inverse map is one-to-many: a position $\mathbf{r}$ may lie in multiple partition regions (if wavefunctions overlap) or uniquely in one partition (if wavefunctions are well-separated).

This bijection ensures that categorical and physical information are related but not redundant.

\subsection{Why Categorical and Physical Cannot Interfere}

Categorical and physical observables act on different Hilbert space factors, so they cannot interfere. The full state is:
\begin{equation}
|\psi\rangle = \sum_{n,\ell,m,s} \int d^3r \, c_{n\ell ms}(\mathbf{r}) |n,\ell,m,s\rangle \otimes |\mathbf{r}\rangle
\end{equation}

Categorical operators act on the sum:
\begin{equation}
\hat{O}_{\text{cat}} |\psi\rangle = \sum_{n,\ell,m,s} f(n,\ell,m,s) \int d^3r \, c_{n\ell ms}(\mathbf{r}) |n,\ell,m,s\rangle \otimes |\mathbf{r}\rangle
\end{equation}

Physical operators act on the integral:
\begin{equation}
\hat{O}_{\text{phys}} |\psi\rangle = \sum_{n,\ell,m,s} \int d^3r \, g(\mathbf{r}) c_{n\ell ms}(\mathbf{r}) |n,\ell,m,s\rangle \otimes |\mathbf{r}\rangle
\end{equation}

Since $f$ and $g$ operate on different indices (discrete vs continuous), they commute.
