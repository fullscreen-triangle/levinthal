\section{Observer Invariance}

\subsection{The Principle of Observer Invariance}

\begin{axiom}[Observer Invariance]
Physical reality is independent of the number of observers and the order in which they perform measurements.
\end{axiom}

This principle is fundamental to science: objective laws cannot depend on subjective observation protocols. It is analogous to Lorentz invariance in special relativity: physical laws are the same in all inertial frames.

\subsection{Formalization}

Let $\hat{O}_A$ and $\hat{O}_B$ be two observables, and consider three observers:
\begin{itemize}
\item \textbf{Observer 1}: Measures $\hat{O}_A$ only, obtains $a_1$.
\item \textbf{Observer 2}: Measures $\hat{O}_B$ only, obtains $b_2$.
\item \textbf{Observer 3}: Measures both $\hat{O}_A$ and $\hat{O}_B$ simultaneously, obtains $(a_3, b_3)$.
\end{itemize}

Observer invariance requires:
\begin{equation}
a_1 = a_3, \quad b_2 = b_3
\end{equation}

Otherwise, the measurement outcomes would depend on whether other observers are present (violating invariance).

\subsection{Violation Analysis}

Suppose $[\hat{O}_A, \hat{O}_B] \neq 0$. Then measuring $\hat{O}_A$ disturbs $\hat{O}_B$. Observer 3, who measures $\hat{O}_A$ first, collapses the state to $|a_3\rangle$, changing the probability distribution of $\hat{O}_B$:
\begin{equation}
P(B = b | A \text{ measured}) \neq P(B = b | A \text{ not measured})
\end{equation}

Therefore, $b_3 \neq b_2$ in general. This means the outcome of measuring $\hat{O}_B$ depends on whether Observer 1 also measured $\hat{O}_A$, violating invariance.

\subsection{Proof Structure}

\begin{theorem}[Commutation from Invariance]
If physical reality is observer-invariant and measurements are reliable, then all spectroscopic observables commute.
\end{theorem}

\begin{proof}
By contradiction. Assume $[\hat{O}_A, \hat{O}_B] \neq 0$ for two spectroscopic observables. Then:
\begin{enumerate}
\item Observer 1 measures $\hat{O}_A$ on system $S$, obtains $a_1$.
\item Observer 2 measures $\hat{O}_B$ on an identically prepared system $S'$, obtains $b_2$.
\item Observer 3 measures both $\hat{O}_A$ and $\hat{O}_B$ on system $S''$, obtains $(a_3, b_3)$.
\end{enumerate}

If measurements are reliable, $a_1 = a_3 = a_{\text{true}}$ (the true value of $\hat{O}_A$ for systems $S, S'', S'''$). Similarly, $b_2 = b_{\text{true}}$.

But if $[\hat{O}_A, \hat{O}_B] \neq 0$, then $b_3 \neq b_2$ because measuring $\hat{O}_A$ disturbed $\hat{O}_B$.

This means Observer 2 and Observer 3 obtain different values for $\hat{O}_B$, contradicting reliability (both should obtain $b_{\text{true}}$) or invariance (the value should not depend on whether $\hat{O}_A$ was also measured).

Therefore, $[\hat{O}_A, \hat{O}_B] = 0$.
\end{proof}

\subsection{Connection to Special Relativity}

Observer invariance in quantum mechanics is analogous to frame invariance in special relativity:

\begin{center}
\begin{tabular}{|l|l|}
\hline
\textbf{Special Relativity} & \textbf{Quantum Mechanics} \\
\hline
Physical laws same in all inertial frames & Physical outcomes same for all observers \\
Lorentz transformation relates frames & Unitary transformation relates observers \\
Spacetime interval invariant & Observable eigenvalues invariant \\
Speed of light constant & Commutation relations constant \\
\hline
\end{tabular}
\end{center}

Just as relativity requires $c$ to be frame-independent, quantum mechanics requires $[\hat{O}_A, \hat{O}_B]$ to be observer-independent. If commutation relations changed depending on who measured, physics would be subjective.

\subsection{Experimental Tests of Invariance}

Observer invariance has been tested in quantum optics through delayed-choice experiments and quantum erasure. Key result: the outcome of measuring an observable does not depend on whether another observer measured a commuting observable earlier, later, or simultaneously.

For spectroscopy specifically, blind inter-laboratory comparisons confirm invariance: multiple labs analyzing the same sample with different techniques obtain consistent results, independent of measurement order or simultaneity.
