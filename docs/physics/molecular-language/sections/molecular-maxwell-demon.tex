\section{Molecular Maxwell Demon System}
\label{sec:mmd}

\subsection{System Architecture}

The Molecular Maxwell Demon (MMD) system integrates the preceding components into a unified framework for database-free peptide identification. The system comprises six layers:

\begin{enumerate}
\item S-Entropy Neural Network (SENN)
\item Empty Dictionary Architecture
\item Categorical Completion Engine
\item Sequence Reconstructor
\item BMD Equivalence Filter
\item Virtual Detector Interface
\end{enumerate}

\subsection{Configuration}

\begin{definition}[MMD Configuration]
The system configuration $\Theta$ specifies:
\begin{align}
\Theta = \{&\beta_S, \; \text{(S-Entropy bandwidth)} \\
&\epsilon_S, \; \text{(dictionary distance threshold)} \\
&\epsilon_m, \; \text{(mass tolerance)} \\
&n_{\max}, \; \text{(maximum gap size)} \\
&c_{\min}, \; \text{(minimum fragment confidence)} \\
&\texttt{cross\_modal}, \; \text{(enable cross-modal validation)} \\
&\texttt{dynamic\_learn}\} \; \text{(enable dictionary learning)}
\end{align}
\end{definition}

\subsection{Spectrum Analysis Pipeline}

\begin{algorithm}[H]
\caption{MMD Spectrum Analysis}
\label{alg:mmd_analysis}
\begin{algorithmic}[1]
\Procedure{AnalyzeSpectrum}{$\mathbf{m}$, $\mathbf{I}$, $m_{\text{prec}}$, $z$, $t_R$}
    \State \textbf{// Step 1: S-Entropy Transformation}
    \State ($\mathbf{S}$, $M$) $\gets$ SEntropyTransform($\mathbf{m}$, $\mathbf{I}$, $m_{\text{prec}}$, $t_R$)

    \State \textbf{// Step 2: BMD Filtering (optional)}
    \If{BMD enabled}
        \State Indices $\gets$ BMDFilter($\mathbf{S}$)
    \Else
        \State Indices $\gets \{1, \ldots, |\mathbf{S}|\}$
    \EndIf

    \State \textbf{// Step 3: Build Fragment Nodes}
    \State $V \gets \varnothing$
    \For{$i \in$ Indices}
        \State $v \gets$ FragmentNode($i$, null, $\mathbf{S}_i$, $\mathbf{m}_i \cdot z$, null, null, 1.0)
        \State $V \gets V \cup \{v\}$
    \EndFor

    \State \textbf{// Step 4: Sequence Reconstruction}
    \State $R \gets$ Reconstruct($V$, $m_{\text{prec}} \cdot z$, $z$)

    \State \textbf{// Step 5: Cross-Modal Validation}
    \If{cross\_modal enabled}
        \State $R \gets$ CrossModalValidate($R$, $\mathbf{m}$, $\mathbf{I}$)
    \EndIf

    \State \textbf{// Step 6: Dynamic Learning}
    \If{dynamic\_learn enabled}
        \State UpdateDictionary($R$, $V$)
    \EndIf

    \State \Return $R$
\EndProcedure
\end{algorithmic}
\end{algorithm}

\subsection{Batch Processing}

For multiple spectra $\{(\mathbf{m}^{(i)}, \mathbf{I}^{(i)}, m_{\text{prec}}^{(i)}, z^{(i)}, t_R^{(i)})\}_{i=1}^N$:

\begin{algorithm}[H]
\caption{MMD Batch Analysis}
\label{alg:mmd_batch}
\begin{algorithmic}[1]
\Procedure{BatchAnalyze}{Spectra}
    \State Results $\gets$ EmptyList()

    \For{$i \in \{1, \ldots, N\}$}
        \State $R_i \gets$ AnalyzeSpectrum(Spectra[$i$])
        \State Results.Append($R_i$)
    \EndFor

    \State \textbf{// Aggregate Statistics}
    \State Sequences $\gets \{R.\mathbf{s} : R \in \text{Results}, R.\mathbf{s} \neq \varnothing\}$
    \State $\bar{c} \gets \frac{1}{N} \sum_i R_i.c$
    \State $N_{\text{high}} \gets |\{R : R.c > 0.7\}|$

    \State \Return (Results, $\bar{c}$, $N_{\text{high}}$)
\EndProcedure
\end{algorithmic}
\end{algorithm}

\subsection{Variance Minimization Principle}

The MMD system operates through variance minimization in S-Entropy space, seeking equilibrium states that correspond to valid molecular identifications.

\begin{theorem}[MMD Equilibrium]
The system state $\xi$ converges to equilibrium:
\begin{equation}
\frac{d\xi}{dt} = -\nabla_\xi \mathcal{V}(\xi)
\end{equation}
where $\mathcal{V}(\xi) = \text{Var}(\mathbf{S}|\xi)$ is the variance of S-Entropy coordinates given system state $\xi$.
\end{theorem}

At equilibrium, $\nabla_\xi \mathcal{V} = 0$, corresponding to minimum-entropy molecular configurations consistent with observed data.

\begin{figure}[htbp]
\centering
\includegraphics[width=0.95\textwidth]{figures/mmd_theory_figure.png}
\caption{\textbf{Maximum Mean Discrepancy (MMD) validates platform independence of S-Entropy framework.}
(\textbf{a}) Visual explanation of MMD metric. Two probability distributions shown
as overlapping filled curves: Distribution 1 (cyan) and Distribution 2 (green).
MMD quantifies the distance between distributions in reproducing kernel Hilbert
space (RKHS), providing a rigorous statistical measure of distributional similarity.
Low MMD value (0.050 in example, yellow annotation box) indicates distributions are
nearly identical, validating that S-Entropy coordinates are invariant across
measurement conditions. This is the theoretical foundation for platform independence.
(\textbf{b}) Comparison of MMD to traditional distribution comparison metrics.
Horizontal bar chart showing sensitivity to distribution differences: Correlation
(0.60, orange), Kolmogorov-Smirnov test (0.70, blue), Chi-square test (0.75, blue),
and MMD (0.95, green). MMD's superior sensitivity (95\% vs. 60-75\% for traditional
metrics) validates its use for rigorous platform independence validation. Traditional
metrics fail to capture multi-dimensional distributional differences that MMD detects.
(\textbf{c}) Platform independence proof via MMD. Overlapping histograms show
S-Entropy value distributions from two different mass spectrometry platforms
(Platform 1 in blue, Platform 2 in green). Near-perfect overlap yields MMD = 0.080,
well below the 0.1 threshold for "excellent" similarity (green annotation box:
"Platform Independent"). This empirically validates that S-Entropy coordinates
are invariant across instruments, ionization methods, and acquisition parameters—a
fundamental requirement for database-free identification. The variance minimization
principle (Section 6.5) ensures this invariance by normalizing physicochemical
properties to [0,1] range.
(\textbf{d}) Categorical equivalence classes in S-Entropy space. Five distinct
clusters shown as scatter plots in (S$_k$, S$_t$) space, each colored differently
and surrounded by dashed ellipse: Class 1 (pink, top-right), Class 2 (orange,
top-left), Class 3 (green, center-left), Class 4 (cyan, bottom-right), Class 5
(purple, center-right). Points within each cluster represent molecules with similar
S-Entropy coordinates, forming categorical equivalence classes (Equation 17).
Spatial separation between clusters (no overlap) enables unambiguous classification
and validates the categorical completion approach (Section 5.3), where gaps can be
filled by selecting amino acids from appropriate equivalence classes.
This figure establishes the theoretical and empirical foundation for platform
independence, a key advantage of the S-Entropy framework over traditional database
methods that require platform-specific spectral libraries. MMD validation (MMD < 0.1)
proves that S-Entropy coordinates are universal molecular descriptors, enabling
zero-shot identification across diverse experimental conditions without retraining
or recalibration.}
\label{fig:mmd_theory}
\end{figure}


\subsection{Cross-Modal Pathway Validation}

\begin{definition}[Cross-Modal Validation]
Given reconstructed sequence $\mathbf{s}$ and observed spectrum $(\mathbf{m}, \mathbf{I})$:
\begin{enumerate}
\item Generate theoretical fragment set $F_{\text{theo}} \gets$ Grammar($\mathbf{s}$)
\item Compute theoretical m/z values $\{m(f) : f \in F_{\text{theo}}\}$
\item Match with observed peaks: $N_{\text{match}} = |\{f : \exists m_j, |m(f) - m_j| \leq \epsilon_m\}|$
\item Validation score: $\text{score} = N_{\text{match}} / |F_{\text{theo}}|$
\end{enumerate}
\end{definition}

\subsection{Dictionary Update Protocol}

\begin{algorithm}[H]
\caption{Dictionary Update}
\label{alg:dict_update}
\begin{algorithmic}[1]
\Procedure{UpdateDictionary}{$R$, $V$}
    \State $N_{\text{novel}} \gets |\{v \in V : c(v) < 0.5\}|$

    \For{$v \in V$ with $c(v) < 0.5$}
        \State $(\text{match}, d) \gets$ Lookup($\mathcal{D}$, $\mathbf{S}(v)$, 1, $\epsilon_S$)
        \If{match = null \textbf{or} $d > \epsilon_S$}
            \State LearnNovel($\mathcal{D}$, $\mathbf{S}(v)$, $m(v)$, 0.5)
        \EndIf
    \EndFor
\EndProcedure
\end{algorithmic}
\end{algorithm}

\subsection{System Output}

The MMD system produces:
\begin{itemize}
\item Reconstructed peptide sequences
\item Per-sequence confidence scores
\item Fragment coverage metrics
\item Gap-filled regions with confidence
\item Cross-modal validation scores
\item Updated dictionary (if learning enabled)
\end{itemize}
