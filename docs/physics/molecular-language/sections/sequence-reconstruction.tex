\section{Categorical Sequence Reconstruction}
\label{sec:sequence-reconstruction}

\subsection{Problem Formulation}

\begin{definition}[Sequence Reconstruction Problem]
Given a set of fragment nodes $V = \{v_1, \ldots, v_n\}$ with S-Entropy coordinates and masses, reconstruct the peptide sequence $\mathbf{s}^* \in \mathcal{A}^+$ that generated these fragments.
\end{definition}

\begin{figure}[htbp]
\centering
\includegraphics[width=0.85\textwidth]{figures/sentropy_3d_trajectory.png}
\caption{\textbf{Three-dimensional peptide trajectories demonstrate sequence-specific
S-Entropy paths.}
Three peptide sequences shown as continuous trajectories through S-Entropy space
(S$_k$, S$_t$, S$_e$): PEPTIDE (red/pink path), SAMPLE (green path), and PROTEIN
(blue path). Each sphere represents one amino acid position, with sphere size
proportional to molecular mass and color indicating peptide identity. Line segments
connect sequential amino acids, forming smooth paths from N-terminus to C-terminus.
Trajectory topology encodes sequence information: PROTEIN (blue) shows high S$_e$
excursion (top of plot, indicating charged residues), SAMPLE (green) follows a
compact path at low S$_t$ (small residues like Glycine), and PEPTIDE (red) traces
an intermediate path. Spatial separation between trajectories demonstrates that
different sequences occupy distinct regions of S-Entropy space, enabling sequence
discrimination without database matching. The continuous, non-intersecting paths
validate the sequence coordinate path formalism (Equation 7) and support the
fragment graph reconstruction approach (Section 3), where observed fragments
constrain the path and categorical completion fills gaps. Smooth trajectory curvature
indicates that S-Entropy coordinates change gradually along sequences, ensuring that
adjacent amino acids have similar S-Entropy values—a key assumption for greedy path
construction (Algorithm in Section 3.6). The three-dimensional visualization reveals
that S-Entropy space has sufficient dimensionality to separate diverse peptide
sequences, validating the tri-dimensional coordinate system (Definition 1).}
\label{fig:3d_trajectories}
\end{figure}

The reconstruction minimizes total S-Entropy subject to mass and grammatical constraints:
\begin{equation}
\mathbf{s}^* = \arg\min_{\mathbf{s} \in \mathcal{A}^+} H_S(\mathbf{s}) \quad \text{s.t.} \quad \mathbf{s} \models \mathcal{G}(V)
\end{equation}
where $\mathbf{s} \models \mathcal{G}(V)$ denotes that sequence $\mathbf{s}$ is consistent with the fragment graph.

\subsection{Gap Region Identification}

\begin{definition}[Gap Region]
A gap region $g$ between consecutive fragments $v_i$ and $v_j$ in the reconstructed path is:
\begin{equation}
g = (v_i, v_j, \Delta m, \mathbf{S}_i, \mathbf{S}_j)
\end{equation}
where $\Delta m = m(v_j) - m(v_i) - m_{\min}$ is the excess mass beyond a single amino acid, and $m_{\min} = \min_{a \in \mathcal{A}} m(a)$.
\end{definition}

A gap is identified when:
\begin{equation}
\Delta m > m_{\min} \Rightarrow \exists \text{ gap between } v_i \text{ and } v_j
\end{equation}

\subsection{Categorical Completion}

\begin{definition}[Categorical Completer]
The categorical completer $\mathcal{K}$ maps gap regions to candidate amino acid sequences:
\begin{equation}
\mathcal{K}: \mathcal{G} \rightarrow 2^{\mathcal{A}^* \times [0,1]}
\end{equation}
returning pairs of candidate sequences with confidence scores.
\end{definition}

\begin{algorithm}[H]
\caption{Categorical Gap Completion}
\label{alg:categorical_completion}
\begin{algorithmic}[1]
\Procedure{FillGap}{$\mathcal{D}$, $g$, $\epsilon_m$}
    \State Candidates $\gets$ EmptyList()
    \State $n_{\max} \gets \lfloor \Delta m(g) / m_{\min} \rfloor + 1$

    \For{$n \in \{1, \ldots, n_{\max}\}$}
        \For{$\mathbf{a} \in \mathcal{A}^n$}
            \State $m_{\mathbf{a}} \gets \sum_{a \in \mathbf{a}} m(a)$
            \If{$|m_{\mathbf{a}} - \Delta m(g)| \leq \epsilon_m$}
                \State $\mathbf{S}_{\mathbf{a}} \gets$ PathMidpoint($\mathbf{a}$)
                \State $d \gets$ InterpolationDistance($\mathbf{S}_{\mathbf{a}}$, $\mathbf{S}_i(g)$, $\mathbf{S}_j(g)$)
                \State $c \gets \exp(-d / \sigma)$
                \State Candidates.Append(($\mathbf{a}$, $c$))
            \EndIf
        \EndFor
    \EndFor

    \State Candidates.SortByConfidence()
    \State \Return Candidates[$0$]
\EndProcedure
\end{algorithmic}
\end{algorithm}

\subsection{Reconstruction Algorithm}

\begin{algorithm}[H]
\caption{Sequence Reconstruction}
\label{alg:sequence_reconstruction}
\begin{algorithmic}[1]
\Procedure{Reconstruct}{$V$, $m_{\text{prec}}$, $z$}
    \State \textbf{Step 1:} $\mathcal{G} \gets$ BuildFragmentGraph($V$, $m_{\text{prec}}$)

    \State \textbf{Step 2:} ManifoldExtracted (implicit in coordinates)

    \State \textbf{Step 3:} Path $\gets$ FindHamiltonianPath($\mathcal{G}$)
    \If{Path = null}
        \State \Return FailedReconstruction
    \EndIf

    \State \textbf{Step 4-5:} (Identified, Gaps) $\gets$ IdentifyFragmentsAndGaps(Path, $\mathcal{G}$)

    \State \textbf{Step 6:} FilledGaps $\gets \{\}$
    \For{$g \in$ Gaps}
        \State FilledGaps[$g$] $\gets$ FillGap($\mathcal{D}$, $g$, $\epsilon_m$)
    \EndFor

    \State \textbf{Step 7:} $\mathbf{s} \gets$ ConcatenateSequence(Path, Identified, FilledGaps)

    \State \textbf{Step 8:} Metrics $\gets$ ComputeMetrics($\mathbf{s}$, $V$, Gaps)

    \State \Return ReconstructionResult($\mathbf{s}$, Metrics)
\EndProcedure
\end{algorithmic}
\end{algorithm}

\subsection{Reconstruction Result}

\begin{definition}[Reconstruction Result]
The reconstruction result is a tuple:
\begin{equation}
R = (\mathbf{s}, c, \phi, G, H, V)
\end{equation}
where:
\begin{itemize}
\item $\mathbf{s}$ is the reconstructed sequence
\item $c \in [0, 1]$ is the overall confidence
\item $\phi \in [0, 1]$ is the fragment coverage
\item $G$ is the list of gap-filled regions
\item $H$ is the total path entropy
\item $V$ is the validation score dictionary
\end{itemize}
\end{definition}

\subsection{Coverage and Confidence Metrics}

\begin{definition}[Fragment Coverage]
\begin{equation}
\phi = \frac{\sum_{v \in \text{Identified}} |\sigma(v)|}{\sum_{v \in \text{Path}} |\sigma(v)| + \sum_{g \in G} |\sigma(g)|}
\end{equation}
where $|\sigma(\cdot)|$ denotes sequence length.
\end{definition}

\begin{definition}[Overall Confidence]
\begin{equation}
c = \frac{1}{|V|} \left( \sum_{v \in \text{Identified}} c(v) + \sum_{g \in G} c(g) \right)
\end{equation}
\end{definition}

\subsection{Validation Scores}

The validation score dictionary includes:
\begin{itemize}
\item \texttt{path\_entropy}: Total S-Entropy of the reconstruction path
\item \texttt{mean\_fragment\_conf}: Mean confidence of identified fragments
\item \texttt{mean\_gap\_conf}: Mean confidence of gap completions
\item \texttt{n\_fragments}: Number of identified fragments
\item \texttt{n\_gaps}: Number of filled gaps
\end{itemize}

\subsection{Cross-Modal Validation}

\begin{definition}[Cross-Modal Match Score]
Given reconstructed sequence $\mathbf{s}$, theoretical fragments $F_{\text{theo}}$ are generated via the molecular grammar. The match score is:
\begin{equation}
\text{score}_{\text{CM}} = \frac{|\{f \in F_{\text{theo}} : \exists v \in V, |m(f) - m(v)| \leq \epsilon_m\}|}{|F_{\text{theo}}|}
\end{equation}
\end{definition}

The final confidence is updated as:
\begin{equation}
c_{\text{final}} = \frac{c + \text{score}_{\text{CM}}}{2}
\end{equation}
