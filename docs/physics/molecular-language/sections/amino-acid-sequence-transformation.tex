\section{S-Entropy Coordinate Transformation for Amino Acids}
\label{sec:amino-acid-transformation}

\subsection{Mathematical Foundation}

We define the amino acid S-Entropy transformation $\phi_{AA}: \mathcal{A} \rightarrow \mathcal{S}^3$, where $\mathcal{A}$ represents the set of 20 standard amino acids and $\mathcal{S}^3 \subset \mathbb{R}^3$ represents the three-dimensional S-Entropy coordinate space.

\begin{figure*}[htbp]
\centering
\includegraphics[width=\textwidth]{figures/molecular_language_atlas.png}
\caption{\textbf{Comprehensive atlas of the categorical amino acid molecular language framework.}
(\textbf{a}) Two-dimensional t-SNE projection of amino acids in S-Entropy space, with
charge state encoded by marker shape (upward triangles = positive charge, circles =
neutral, downward triangles = negative charge). Marker size represents molecular mass,
and color indicates hydrophobicity (viridis colormap: yellow = hydrophobic, blue =
hydrophilic). Spatial clustering reveals natural organization: charged residues (R, K,
D, E) cluster at left, hydrophobic residues (F, L, V, I) at right, demonstrating that
S-Entropy coordinates capture physicochemical similarity. Legend shows charge categories.
(\textbf{b}) Periodic table organization of 20 standard amino acids arranged by
hydrophobicity (x-axis, blue to red gradient) and other physicochemical properties.
Each cell displays single-letter code and molecular mass. Color intensity reflects
hydrophobicity scale from -4 (hydrophilic, blue) to +4 (hydrophobic, red). This
organization mirrors the spatial clustering in panel (a), validating the S-Entropy
transformation.
(\textbf{c}) Fragment position versus peptide relative position for three example
sequences (PEPTIDE, SEQUENCE, PROTEIN). Triangles represent fragment observations,
colored by ion type (blue = b-ions, red = y-ions). Y-axis shows peptide identity,
x-axis shows normalized position (0 = N-terminus, 1 = C-terminus). Diagonal pattern
indicates sequential fragment coverage, with gaps representing missing fragments that
require categorical completion (Section 5.3).
(\textbf{d}) Peptide trajectories through S-Entropy space. Line plot showing S-Time
(y-axis) versus S-Knowledge (x-axis) for three peptides (PROTEIN in blue, SAMPLE in
green, PEPTIDE in red). Each point represents one amino acid position, connected by
lines showing sequential progression. Distinct trajectory shapes encode sequence
identity, with labeled points (e.g., "PROTEIN", "SAMPLE") marking specific positions.
Smooth paths validate that sequences form continuous trajectories (Equation 7).
(\textbf{e}) Sequence entropy versus sequence complexity scatter plot. Three peptides
shown as labeled points with rectangular borders: PEPTIDE (red, high entropy 2.24,
high complexity 0.499), SAMPLE (orange, moderate values), PROTEIN (blue, highest
entropy 2.8). Color gradient from yellow (high density) to red (low density) indicates
local density of sequences in entropy-complexity space. Demonstrates that different
sequences occupy distinct regions, enabling discrimination.
(\textbf{f}) Amino acid similarity matrix (heatmap) showing pairwise Euclidean
distances in S-Entropy space. Rows and columns represent amino acids (single-letter
codes). Color scale from black (distance = 0, identical) through purple, pink, yellow
to white (distance = 1.0, maximally different). Diagonal is black (self-similarity).
Block structure reveals equivalence classes: hydrophobic cluster (F, L, I, V, M),
charged cluster (K, R, D, E), polar cluster (S, T, N, Q). This matrix defines
equivalence classes (Equation 17) used for categorical completion.
(\textbf{g}) Fragment type distribution shown as radial sector plot. Large blue sector
labeled "b" represents b-ion dominance (>90\% of observed fragments). Small white
wedge indicates other ion types. This distribution validates focusing reconstruction
on b-ion series (Section 2.2), as they provide most sequence information.
(\textbf{h}) Categorical completion example showing three amino acids (orange, blue,
green circles) connected by gray lines in S-Entropy space. Represents gap-filling
process where missing amino acids are inferred from S-Entropy proximity to observed
fragments.}
\label{fig:molecular_atlas}
\end{figure*}

\begin{definition}[S-Entropy Coordinate Space]
The S-Entropy coordinate space is defined as:
\begin{equation}
\mathcal{S}^3 = \{(S_k, S_t, S_e) \in \mathbb{R}^3 : S_k, S_t, S_e \in [0, 1]\}
\end{equation}
where $S_k$ denotes the knowledge dimension, $S_t$ denotes the time dimension, and $S_e$ denotes the entropy dimension.
\end{definition}

\subsection{Physicochemical Property Mapping}

The transformation $\phi_{AA}$ maps amino acid physicochemical properties to S-Entropy coordinates through the following assignments:

\begin{definition}[Amino Acid S-Entropy Transformation]
For amino acid $a \in \mathcal{A}$ with hydrophobicity $H(a)$, van der Waals volume $V(a)$, charge $Q(a)$, and polarity $P(a)$, the S-Entropy coordinates are:
\begin{align}
S_k(a) &= \frac{H(a) - H_{\min}}{H_{\max} - H_{\min}} \\
S_t(a) &= \frac{V(a)}{V_{\max}} \\
S_e(a) &= \frac{|Q(a)| + \mathbf{1}_{P(a)}}{2}
\end{align}
where $H_{\min} = -4.5$, $H_{\max} = 4.5$ correspond to the Kyte-Doolittle scale bounds, $V_{\max} = 250$ \AA$^3$, and $\mathbf{1}_{P(a)}$ is the indicator function for polar residues.
\end{definition}

The knowledge dimension $S_k$ captures information content through hydrophobicity, which correlates with membrane interaction propensity and protein folding energetics. The time dimension $S_t$ encodes molecular size through van der Waals volume. The entropy dimension $S_e$ quantifies electrostatic complexity through the combined effect of formal charge and hydrogen-bonding capacity.

\subsection{Standard Amino Acid Coordinates}

Table~\ref{tab:amino_acid_coords} presents the S-Entropy coordinates for the 20 standard amino acids computed via the transformation $\phi_{AA}$.

\begin{table}[H]
\centering
\begin{tabular}{lcccccc}
\toprule
Amino Acid & Symbol & Mass (Da) & $S_k$ & $S_t$ & $S_e$ \\
\midrule
Alanine & A & 71.037 & 0.700 & 0.268 & 0.000 \\
Arginine & R & 156.101 & 0.000 & 0.592 & 1.000 \\
Asparagine & N & 114.043 & 0.111 & 0.384 & 0.500 \\
Aspartic acid & D & 115.027 & 0.111 & 0.364 & 1.000 \\
Cysteine & C & 103.009 & 0.778 & 0.344 & 0.500 \\
Glutamine & Q & 128.059 & 0.111 & 0.456 & 0.500 \\
Glutamic acid & E & 129.043 & 0.111 & 0.436 & 1.000 \\
Glycine & G & 57.021 & 0.456 & 0.192 & 0.000 \\
Histidine & H & 137.059 & 0.144 & 0.472 & 0.500 \\
Isoleucine & I & 113.084 & 1.000 & 0.496 & 0.000 \\
Leucine & L & 113.084 & 0.922 & 0.496 & 0.000 \\
Lysine & K & 128.095 & 0.067 & 0.540 & 1.000 \\
Methionine & M & 131.040 & 0.711 & 0.496 & 0.000 \\
Phenylalanine & F & 147.068 & 0.811 & 0.540 & 0.000 \\
Proline & P & 97.053 & 0.322 & 0.360 & 0.000 \\
Serine & S & 87.032 & 0.411 & 0.292 & 0.500 \\
Threonine & T & 101.048 & 0.422 & 0.372 & 0.500 \\
Tryptophan & W & 186.079 & 0.400 & 0.652 & 0.000 \\
Tyrosine & Y & 163.063 & 0.356 & 0.564 & 0.500 \\
Valine & V & 99.068 & 0.967 & 0.420 & 0.000 \\
\bottomrule
\end{tabular}
\caption{S-Entropy coordinates for standard amino acids.}
\label{tab:amino_acid_coords}
\end{table}

\subsection{Post-Translational Modification Shifts}

Post-translational modifications (PTMs) are represented as affine transformations in S-Entropy space.

\begin{definition}[PTM S-Entropy Shift]
For a post-translational modification $\tau$ applied to amino acid $a$, the modified coordinates are:
\begin{equation}
\phi_{AA}(a^\tau) = \phi_{AA}(a) + \Delta\mathbf{S}_\tau
\end{equation}
where $\Delta\mathbf{S}_\tau = (\Delta S_k, \Delta S_t, \Delta S_e)$ is the PTM-specific shift vector.
\end{definition}

The shift components are computed as:
\begin{align}
\Delta S_k &= 0.2 \cdot \tanh\left(\frac{\Delta m_\tau}{100}\right) \\
\Delta S_t &= \frac{\Delta m_\tau}{200} \\
\Delta S_e &= \text{sgn}(\Delta m_\tau) \cdot 0.1
\end{align}
where $\Delta m_\tau$ is the mass shift induced by the modification.

\subsection{Sequence Coordinate Path}

\begin{definition}[Peptide S-Entropy Path]
For a peptide sequence $\mathbf{s} = (s_1, s_2, \ldots, s_n)$ where $s_i \in \mathcal{A}$, the S-Entropy coordinate path is:
\begin{equation}
\mathbf{P}(\mathbf{s}) = (\phi_{AA}(s_1), \phi_{AA}(s_2), \ldots, \phi_{AA}(s_n))
\end{equation}
forming a trajectory through $\mathcal{S}^3$.
\end{definition}

The cumulative path displacement is defined as:
\begin{equation}
\mathbf{D}(\mathbf{s}) = \sum_{i=1}^n \phi_{AA}(s_i)
\end{equation}

\begin{figure}[htbp]
\centering
\includegraphics[width=0.95\textwidth]{figures/Figure1_SEntropy_Space.png}
\caption{\textbf{Amino acid representation in tri-dimensional S-Entropy coordinate space.}
(\textbf{a}) Three-dimensional scatter plot showing all 20 standard amino acids in
S-Entropy space (S$_k$, S$_t$, S$_e$). Amino acids are colored by chemical category:
charged (orange), polar (teal), hydrophobic (blue), aromatic (green), and special
(yellow). Marker shapes encode charge state: upward triangles (positive: K, R, H),
circles (neutral), downward triangles (negative: D, E). Single-letter codes label
each amino acid. Spatial separation between categories demonstrates that physicochemical
properties map to distinct S-Entropy coordinates, enabling categorical identification.
(\textbf{b}) Two-dimensional projection of amino acids onto S-Knowledge (S$_k$) versus
S-Time (S$_t$) plane, with points colored by S-Entropy (S$_e$) using viridis colormap
(yellow = high entropy, purple = low entropy). Hydrophobic residues (F, W, I, L, V, M)
cluster at high S$_k$ values (right), while charged residues (K, R, D, E) cluster at
low S$_k$ values (left). Molecular volume correlates with S$_t$ (y-axis).
(\textbf{c}) S-Entropy distribution across amino acids in (S$_k$, S$_e$) space.
Charged residues (K, R at top) show high S$_e$ values (electrostatic complexity),
while special residues (G, P, C at bottom) show low S$_e$ values. Hydrophobic residues
occupy intermediate S$_e$ range.
(\textbf{d}) Pearson correlation analysis between physicochemical properties and
S-Entropy dimensions. Hydrophobicity strongly correlates with S$_k$ (r = 1.000),
molecular volume with S$_t$ (r = 1.000), and charge with S$_e$ (r = 0.815). High
correlations validate the coordinate transformation defined in Equations 2-4.
This comprehensive visualization establishes that the S-Entropy transformation
(Definition 1) maps amino acids to a well-structured coordinate space where
physicochemical similarity corresponds to spatial proximity, forming the foundation
for database-free categorical identification.}
\label{fig:sentropy_space}
\end{figure}

\subsection{Sequence S-Entropy}

\begin{definition}[Sequence S-Entropy]
The sequence S-Entropy measures the information content of the coordinate path:
\begin{equation}
H_S(\mathbf{s}) = -\sum_{\mathbf{c} \in \mathcal{B}} p(\mathbf{c}) \log_2 p(\mathbf{c})
\end{equation}
where $\mathcal{B}$ is a binning of $\mathcal{S}^3$ into discrete states and $p(\mathbf{c})$ is the empirical probability of coordinate path elements falling in bin $\mathbf{c}$.
\end{definition}

\begin{definition}[Sequence Complexity]
The sequence complexity score combines Shannon entropy with a repetition penalty:
\begin{equation}
C(\mathbf{s}) = \frac{H(\mathbf{s})}{H_{\max}} \cdot \left(1 - \frac{\ell_{\max}(\mathbf{s})}{n}\right)
\end{equation}
where $H(\mathbf{s})$ is the Shannon entropy of amino acid frequencies, $H_{\max} = \log_2(20)$ is the maximum entropy for 20 amino acids, and $\ell_{\max}(\mathbf{s})$ is the length of the longest repeating substring.
\end{definition}
