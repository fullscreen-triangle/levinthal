\section{Fragment Graph Construction}
\label{sec:fragment-graph}

\subsection{Graph Formalism}

We construct a directed graph $\mathcal{G} = (V, E)$ where vertices represent observed fragments and edges represent sequential relationships consistent with peptide bond cleavage patterns.

\begin{definition}[Fragment Node]
A fragment node $v \in V$ is a tuple:
\begin{equation}
v = (id, \sigma, \mathbf{S}, m, \tau, k, c)
\end{equation}
where:
\begin{itemize}
\item $id$ is a unique identifier
\item $\sigma \in \mathcal{A}^*$ is the partial sequence (if identified)
\item $\mathbf{S} \in \mathcal{S}^3$ is the S-Entropy coordinate vector
\item $m \in \mathbb{R}^+$ is the fragment mass
\item $\tau \in \{b, y, a, c, x, z, \varnothing\}$ is the ion type
\item $k \in \mathbb{Z}^+$ is the sequence position (if known)
\item $c \in [0, 1]$ is the identification confidence
\end{itemize}
\end{definition}


\begin{figure}[htbp]
\centering
\includegraphics[width=0.95\textwidth]{figures/Figure2_Sequence_Trajectories.png}
\caption{\textbf{Peptide sequences as continuous paths in S-Entropy space.}
(\textbf{a}) Three-dimensional trajectory of the peptide "PEPTIDE" through S-Entropy
space (S$_k$, S$_t$, S$_e$). Each sphere represents one amino acid position, connected
by line segments showing sequential progression from N-terminus to C-terminus. Spheres
are colored by amino acid type (orange for charged, blue for hydrophobic, etc.).
The smooth, continuous path demonstrates that peptide sequences form coherent
trajectories in S-Entropy space, validating the sequence coordinate path concept
(Equation 7).
(\textbf{b}) Three-dimensional trajectory of the peptide "SAMPLE" in S-Entropy space,
showing a different path topology. The distinct trajectory shape reflects the unique
amino acid composition and sequence order, demonstrating that different peptides
occupy different regions of S-Entropy space.
(\textbf{c}) Sequence entropy and complexity metrics for both peptides. Bar chart
comparing sequence entropy (orange, calculated via Equation 8) and complexity
(blue, scaled ×5 for visualization). PEPTIDE shows higher entropy (2.24) and
complexity (2.58) than SAMPLE (0.444 and 0.499 respectively), reflecting greater
amino acid diversity and physicochemical heterogeneity.
(\textbf{d}) S-Entropy magnitude evolution along sequence positions. Line plot
showing how total S-Entropy magnitude (Equation 9) varies across positions for
PEPTIDE (orange) and SAMPLE (blue). Peaks correspond to amino acids with extreme
physicochemical properties (e.g., charged residues), while valleys indicate neutral
residues. The distinct patterns enable sequence discrimination.
This figure demonstrates that peptide sequences trace unique, continuous paths
through S-Entropy space, with path topology encoding sequence identity. The smooth
trajectories validate using S-Entropy coordinates for sequence reconstruction, as
fragments from the same peptide will lie on the same continuous path.}
\label{fig:sequence_trajectories}
\end{figure}



\subsection{Edge Construction}

\begin{definition}[Sequential Edge]
An edge $e = (v_i, v_j) \in E$ connects fragments if they satisfy the sequential relationship constraint:
\begin{equation}
\exists a \in \mathcal{A}: |m(v_j) - m(v_i) - m(a)| \leq \epsilon_m
\end{equation}
where $\epsilon_m$ is the mass tolerance (typically 0.5 Da).
\end{definition}

\begin{definition}[Edge Weight]
The edge weight incorporates S-Entropy similarity:
\begin{equation}
w(v_i, v_j) = \exp\left(-\frac{\|\mathbf{S}(v_i) - \mathbf{S}(v_j)\|}{\sigma_S}\right)
\end{equation}
where $\sigma_S$ is a bandwidth parameter (default 0.3).
\end{definition}

\subsection{Graph Construction Algorithm}

\begin{algorithm}[H]
\caption{Fragment Graph Construction}
\label{alg:graph_construction}
\begin{algorithmic}[1]
\Procedure{BuildFragmentGraph}{FragmentList, $\epsilon_m$}
    \State $\mathcal{G} \gets$ EmptyDirectedGraph()
    \State $\mathcal{M}_{AA} \gets \{m(a) : a \in \mathcal{A}\}$

    \For{$v \in$ FragmentList}
        \State $\mathcal{G}$.AddNode($v$)
    \EndFor

    \For{$v_i \in \mathcal{G}$.Nodes()}
        \For{$v_j \in \mathcal{G}$.Nodes(), $v_j \neq v_i$}
            \State $\Delta m \gets m(v_j) - m(v_i)$
            \For{$m_a \in \mathcal{M}_{AA}$}
                \If{$|\Delta m - m_a| \leq \epsilon_m$}
                    \State $w \gets$ ComputeSEntropySimilarity($v_i$, $v_j$)
                    \If{$m(v_j) > m(v_i)$}
                        \State $\mathcal{G}$.AddEdge($v_i$, $v_j$, $w$)
                    \EndIf
                    \State \textbf{break}
                \EndIf
            \EndFor
        \EndFor
    \EndFor

    \State \Return $\mathcal{G}$
\EndProcedure
\end{algorithmic}
\end{algorithm}

\subsection{S-Entropy Magnitude}

\begin{definition}[Fragment Entropy]
The S-Entropy magnitude for a fragment node is:
\begin{equation}
H(v) = \|\mathbf{S}(v)\|_2 = \sqrt{S_k^2 + S_t^2 + S_e^2}
\end{equation}
\end{definition}

\subsection{Path Finding}

\begin{definition}[Hamiltonian Path Problem]
The sequence reconstruction problem reduces to finding a Hamiltonian path through $\mathcal{G}$ that minimizes total S-Entropy:
\begin{equation}
\mathbf{p}^* = \arg\min_{\mathbf{p} \in \mathcal{H}(\mathcal{G})} \sum_{v \in \mathbf{p}} H(v)
\end{equation}
where $\mathcal{H}(\mathcal{G})$ is the set of Hamiltonian paths in $\mathcal{G}$.
\end{definition}

For directed acyclic graphs, the longest path can be computed via dynamic programming in $O(|V| + |E|)$ time.

\begin{algorithm}[H]
\caption{Longest Path in DAG}
\label{alg:longest_path}
\begin{algorithmic}[1]
\Procedure{FindLongestPath}{$\mathcal{G}$}
    \State $d[v] \gets -\infty$ for all $v \in V$
    \State $\pi[v] \gets$ null for all $v \in V$
    \State TopologicalOrder $\gets$ TopologicalSort($\mathcal{G}$)

    \For{$v$ with in-degree = 0}
        \State $d[v] \gets 0$
    \EndFor

    \For{$v \in$ TopologicalOrder}
        \For{$u \in$ Successors($v$)}
            \If{$d[v] + w(v, u) > d[u]$}
                \State $d[u] \gets d[v] + w(v, u)$
                \State $\pi[u] \gets v$
            \EndIf
        \EndFor
    \EndFor

    \State $v_{\text{end}} \gets \arg\max_v d[v]$
    \State Path $\gets$ ReconstructPath($\pi$, $v_{\text{end}}$)
    \State \Return Path
\EndProcedure
\end{algorithmic}
\end{algorithm}

\subsection{Greedy Path Construction}

When the graph contains cycles, we employ greedy path construction:

\begin{algorithm}[H]
\caption{Greedy Path Construction}
\label{alg:greedy_path}
\begin{algorithmic}[1]
\Procedure{GreedyPath}{$\mathcal{G}$}
    \State $v_0 \gets \arg\min_{v \in V} m(v)$
    \State Path $\gets [v_0]$
    \State Visited $\gets \{v_0\}$
    \State $v_c \gets v_0$

    \While{$|$Visited$| < |V|$}
        \State $v_{\text{next}} \gets$ null
        \State $w_{\text{best}} \gets -\infty$
        \For{$u \in$ Successors($v_c$), $u \notin$ Visited}
            \If{$w(v_c, u) > w_{\text{best}}$}
                \State $w_{\text{best}} \gets w(v_c, u)$
                \State $v_{\text{next}} \gets u$
            \EndIf
        \EndFor
        \If{$v_{\text{next}} =$ null}
            \State \textbf{break}
        \EndIf
        \State Path.Append($v_{\text{next}}$)
        \State Visited.Add($v_{\text{next}}$)
        \State $v_c \gets v_{\text{next}}$
    \EndWhile

    \State \Return Path
\EndProcedure
\end{algorithmic}
\end{algorithm}

\subsection{Path Entropy Calculation}

\begin{definition}[Total Path Entropy]
For a path $\mathbf{p} = (v_1, v_2, \ldots, v_\ell)$:
\begin{equation}
H_{\text{total}}(\mathbf{p}) = \sum_{i=1}^\ell H(v_i)
\end{equation}
\end{definition}

Lower path entropy indicates greater structural organization and higher reconstruction confidence.
