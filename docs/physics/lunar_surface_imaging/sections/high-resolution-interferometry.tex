\section{High-Resolution Interferometry from First Principles}
\label{sec:interferometry}

\subsection{Telescopes as Partition Signature Collectors}

Telescopes are not merely "light buckets" but devices that collect and combine partition signatures from extended sources.

\begin{definition}[Telescope as Partition Collector]
\label{def:telescope}
A telescope is a bounded aperture of diameter $D$ that:
\begin{enumerate}
    \item Collects photons (partition signature carriers) from solid angle $\Omega$
    \item Focuses collected signatures onto detector partition space (focal plane)
    \item Achieves partition depth $n_{\text{telescope}} \sim D/\lambda$ at wavelength $\lambda$
\end{enumerate}
\end{definition}

\begin{theorem}[Single-Aperture Partition Depth]
\label{thm:single_aperture_depth}
A circular aperture of diameter $D$ observing at wavelength $\lambda$ achieves partition depth:
\begin{equation}
n_{\text{single}} = \frac{D}{\lambda}
\end{equation}
This is the number of distinguishable partition categories across the aperture.
\end{theorem}

\begin{proof}
The aperture subdivides the incoming wavefront (oscillatory field) into regions of size $\sim \lambda$ (characteristic oscillation scale). The number of such regions across diameter $D$ is:
\begin{equation}
n = \frac{D}{\lambda}
\end{equation}

Each region can be in a different partition state (phase, amplitude), giving $n$ distinguishable categories. This is the partition depth of the aperture.

From partition depth, angular resolution follows (Theorem~\ref{thm:resolution_limit_derived}):
\begin{equation}
\delta\theta = \frac{1}{n} = \frac{\lambda}{D}
\end{equation}

This is the Rayleigh diffraction limit, derived here from partition geometry without invoking wave optics as an independent axiom. Diffraction is not a fundamental phenomenon but an emergent consequence of partition depth limitations.

\textbf{Example:} Hubble Space Telescope with $D = 2.4$ m at $\lambda = 550$ nm:
\begin{equation}
n_{\text{HST}} = \frac{2.4}{550 \times 10^{-9}} \approx 4.4 \times 10^6
\end{equation}

This partition depth corresponds to angular resolution:
\begin{equation}
\delta\theta_{\text{HST}} = \frac{550 \times 10^{-9}}{2.4} \approx 2.3 \times 10^{-7} \text{ rad} \approx 0.047 \text{ arcsec}
\end{equation}
\end{proof}

\begin{remark}
This establishes that telescope resolution is fundamentally a partition depth phenomenon, not a wave diffraction phenomenon. The wave description is an effective theory valid when partition depth is the limiting factor. At higher partition depths (achieved through interferometry or virtual imaging), wave optics breaks down and categorical measurement becomes necessary.
\end{remark}

\subsection{Interferometry as Multi-Modal Partition Combination}

Combining multiple telescopes increases effective partition depth through categorical composition.

\begin{theorem}[Interferometric Partition Depth Enhancement]
\label{thm:interferometric_depth}
An interferometer with $K$ apertures of individual partition depths $\{n_k\}$ arranged with maximum baseline $B$ achieves effective partition depth:
\begin{equation}
n_{\text{eff}} = \sqrt{\sum_{k=1}^K n_k^2}
\end{equation}
For apertures with $n_k \approx D/\lambda$ separated by baseline $B \gg D$, this simplifies to:
\begin{equation}
n_{\text{eff}} \approx \frac{B}{\lambda}
\end{equation}
\end{theorem}

\begin{proof}
Each aperture provides partition coordinates up to depth $n_k$. From Theorem~\ref{thm:capacity}, each aperture can distinguish $2n_k^2$ states.

Independent apertures measure orthogonal partition information—different spatial sampling of the target partition structure. The combined categorical space has total capacity:
\begin{equation}
C_{\text{total}} = \sum_{k=1}^K 2n_k^2
\end{equation}

The effective partition depth $n_{\text{eff}}$ satisfying $2n_{\text{eff}}^2 = \sum_k 2n_k^2$ is:
\begin{equation}
n_{\text{eff}} = \sqrt{\sum_{k=1}^K n_k^2}
\end{equation}

For $K$ identical apertures with $n_k = n$:
\begin{equation}
n_{\text{eff}} = \sqrt{K n^2} = n\sqrt{K}
\end{equation}

For apertures separated by baseline $B$, the longest baseline dominates partition depth. The interferometer effectively acts as a single aperture of diameter $B$ (with gaps), giving:
\begin{equation}
n_{\text{eff}} \approx \frac{B}{\lambda}
\end{equation}

Angular resolution improves to:
\begin{equation}
\delta\theta_{\text{interferometric}} = \frac{\lambda}{B}
\end{equation}

This is the fundamental principle of interferometry: baseline $B$ replaces aperture diameter $D$ in determining resolution.
\end{proof}

\begin{example}[Lunar Interferometry]
\label{ex:lunar_interferometry}
Consider two telescopes with $D = 10$ m separated by baseline $B = 10$ km observing the Moon at $\lambda = 550$ nm.

\textbf{Single-aperture partition depth:}
\begin{equation}
n_{\text{single}} = \frac{D}{\lambda} = \frac{10}{550 \times 10^{-9}} \approx 1.82 \times 10^7
\end{equation}

\textbf{Interferometric partition depth:}
\begin{equation}
n_{\text{interferometric}} = \frac{B}{\lambda} = \frac{10^4}{550 \times 10^{-9}} \approx 1.82 \times 10^{10}
\end{equation}

Enhancement factor: $n_{\text{interferometric}}/n_{\text{single}} = B/D = 1000$.

\textbf{Single-aperture resolution at Moon distance} ($r = 3.844 \times 10^8$ m):
\begin{equation}
\delta x_{\text{single}} = \frac{\lambda r}{D} = \frac{550 \times 10^{-9} \times 3.844 \times 10^8}{10} \approx 21 \text{ m}
\end{equation}

\textbf{Interferometric resolution:}
\begin{equation}
\delta x_{\text{interfero}} = \frac{\lambda r}{B} = \frac{550 \times 10^{-9} \times 3.844 \times 10^8}{10^4} \approx 0.021 \text{ m} = 2.1 \text{ cm}
\end{equation}

At this resolution:
\begin{itemize}
    \item Apollo lunar modules (width $\sim 9$ m): Clearly resolved ($9 \text{ m} \gg 2.1 \text{ cm}$)
    \item Apollo flags (width $\sim 0.9$ m): Resolved ($0.9 \text{ m} = 90 \text{ cm} \gg 2.1 \text{ cm}$)
    \item Footprints (width $\sim 0.3$ m): Marginally resolved ($0.3 \text{ m} = 30 \text{ cm} > 2.1 \text{ cm}$)
    \item Bootprint treads (spacing $\sim 1$ cm): Unresolved ($1 \text{ cm} < 2.1 \text{ cm}$)
\end{itemize}

A 10 km baseline interferometer can resolve Apollo landing site artifacts at the meter scale but not fine details like bootprint treads.
\end{example}

\begin{remark}
Physical interferometry with $B = 10$ km is technically challenging but feasible:
\begin{itemize}
    \item Very Large Telescope Interferometer (VLTI): $B \sim 200$ m
    \item Very Long Baseline Array (VLBA, radio): $B \sim 8000$ km
    \item Space-based interferometry (proposed): $B \sim 100$ km--$1000$ km
\end{itemize}
However, optical interferometry at $B > 1$ km faces severe challenges (atmospheric turbulence, vibration, thermal expansion). This motivates virtual interferometry.
\end{remark}

\subsection{Virtual Super-Resolution via Information Catalysis}

Beyond physical interferometry, virtual imaging achieves resolution enhancement through partition signature morphisms and information catalysis.

\begin{theorem}[Virtual Resolution Enhancement]
\label{thm:virtual_resolution}
Information catalysts (Definition~\ref{def:info_catalyst}) create intermediate partition stages:
\begin{equation}
\Sigma_{\text{observed}} \xrightarrow{C_1} \Sigma_1 \xrightarrow{C_2} \Sigma_2 \xrightarrow{C_3} \cdots \xrightarrow{C_K} \Sigma_{\text{fine structure}}
\end{equation}
reducing categorical distance and enabling sub-pixel structure inference without additional photon collection.
\end{theorem}

\begin{proof}
Physical observations provide partition signatures at resolution $\delta x_{\text{physical}}$ (limited by aperture size, baseline, or detector pixel size). However, these signatures encode information about finer structures through:

\textbf{Catalyst 1 - Surface texture priors:} Known regolith grain statistics constrain sub-resolution structure. Lunar regolith has characteristic grain size distribution ($\sim 10$--$100$ μm), packing density ($\sim 50\%$), and roughness statistics. These priors propagate partition information from coarse to fine scales:
\begin{equation}
\Sigma_{\text{coarse}} \xrightarrow{\text{texture prior}} \Sigma_{\text{grain level}}
\end{equation}

\textbf{Catalyst 2 - Conservation laws:} Mass, charge, and energy continuity propagate across resolution boundaries. A coarse-resolution measurement of mass distribution constrains fine-resolution structure through conservation:
\begin{equation}
\Sigma_{\text{grain}} \xrightarrow{\text{conservation}} \Sigma_{\text{continuous}}
\end{equation}

\textbf{Catalyst 3 - Phase-lock network continuity:} Chemical bonding patterns extend coherently across scales. Atomic-scale phase-lock networks (covalent bonds, Van der Waals forces) create mesoscale structure (grain boundaries, crystal domains) that propagates to macroscale (surface features):
\begin{equation}
\Sigma_{\text{continuous}} \xrightarrow{\text{network}} \Sigma_{\text{molecular}}
\end{equation}

Each catalyst reduces categorical distance $d_{\text{cat}}$ (Definition~\ref{def:categorical_distance}), making finer partition structures accessible through morphism chains even without additional photon collection.

Effective resolution after $K$ catalysts:
\begin{equation}
\delta x_{\text{virtual}} \approx \frac{\delta x_{\text{physical}}}{\prod_{k=1}^K \gamma_k}
\end{equation}

where $\gamma_k$ is the enhancement factor of catalyst $k$. Typical values: $\gamma \sim 2$--$5$ per stage (depending on prior strength and data quality).

The enhancement is not arbitrary but constrained by:
\begin{itemize}
    \item \textbf{Information content}: Cannot extract more information than present in data
    \item \textbf{Prior validity}: Priors must accurately reflect target structure
    \item \textbf{Morphism availability}: Catalytic pathways must exist in categorical space
\end{itemize}

Virtual resolution enhancement is fundamentally different from interpolation or upsampling. It accesses categorical information present in the data but not directly observable, using catalysts to bridge the categorical distance.
\end{proof}

\begin{corollary}[Apollo Site Virtual Imaging]
\label{cor:apollo_virtual}
Starting from interferometric resolution $\delta x_{\text{interfero}} = 2.1$ cm (Example~\ref{ex:lunar_interferometry}), applying three information catalysts with $\gamma = 3$ each:
\begin{equation}
\delta x_{\text{virtual}} = \frac{2.1 \text{ cm}}{3^3} = \frac{2.1 \text{ cm}}{27} \approx 0.078 \text{ cm} = 0.78 \text{ mm}
\end{equation}
\end{corollary}

\begin{proof}
Three catalysts:
\begin{enumerate}
    \item \textbf{Regolith texture prior} ($\gamma_1 = 3$): Grain size distribution, packing statistics
    \item \textbf{Thermal continuity} ($\gamma_2 = 3$): Temperature gradient constraints from infrared observations
    \item \textbf{Phase-lock network} ($\gamma_3 = 3$): Atomic-scale bonding patterns
\end{enumerate}

Combined enhancement: $\gamma_{\text{total}} = 3 \times 3 \times 3 = 27$.

At $\delta x_{\text{virtual}} \approx 0.78$ mm resolution:
\begin{itemize}
    \item \textbf{Flag fabric texture}: Weave pattern ($\sim 1$ mm spacing) resolved
    \item \textbf{Footprint treads}: Boot sole pattern ($\sim$ few mm features) distinguished
    \item \textbf{Equipment details}: Bolt heads ($\sim 5$ mm), cable connectors ($\sim 1$ cm) visible
    \item \textbf{Serial numbers}: Potentially readable if lettering $\gtrsim 1$ cm (marginal)
\end{itemize}

This resolution exceeds the best Lunar Reconnaissance Orbiter (LRO) images ($\sim 0.5$ m resolution from 50 km orbit), achieved from Earth (384,400 km distance) through categorical measurement rather than photonic imaging.
\end{proof}

\begin{remark}
Virtual resolution enhancement is not "making up data" but accessing categorical information already present in the physical observations. The catalysts do not add information but reduce the categorical distance to existing fine-structure partition signatures. This is analogous to solving an inverse problem: the coarse data constrain the fine structure through physical laws (conservation, continuity, network coherence).
\end{remark}

\begin{figure}[htbp]
\centering
\includegraphics[width=\textwidth]{figures/section_8_validation.png}
\caption{\textbf{Virtual super-resolution demonstration: achieving 27$\times$ resolution enhancement beyond interferometry through categorical partition imaging.}
\textbf{(A)} Partition depth enhancement showing $n_{\text{eff}} = D/\lambda + B/\lambda$ progression. Bar chart comparing partition depths for different observational methods. Blue bar: single aperture (10~m diameter), $n_{\text{eff}} = 1.82 \times 10^7$ (partition depth limited by aperture size). Green bar: interferometer (10~km baseline), $n_{\text{eff}} = 1.82 \times 10^{10}$ (1000$\times$ enhancement from baseline extension). 
\textbf{(B)} Resolution progression showing physical $\to$ interferometric $\to$ virtual enhancement. Three bars showing resolution at lunar distance (384,400~km). Red bar: single telescope (10~m), resolution 21.142~m (flag width 0.9~m shown as blue dashed line—flag NOT visible, far below resolution limit). Orange bar: interferometer (10~km baseline), resolution 0.021~m = 21~mm (flag NOW visible, resolution 40$\times$ better than flag size). Green bar: virtual imaging (3 catalysts), resolution 0.78~mm (flag details visible, resolution 27$\times$ better than interferometry, 1150$\times$ better than flag size). Vertical axis: resolution $\delta x$ at Moon (logarithmic scale, $10^{-3}$--$10^1$~m). 
\textbf{(C)} Spectral partition mapping $I(\lambda) \sim \lambda_{\text{ref}}/\lambda$ showing wavelength-dependent partition coordinates. Bar chart showing angular partition coordinate $\ell$ for different wavelengths. Purple bar: UV (300~nm), $\ell \sim 175{,}000$ (highest angular complexity). Green bar: visible (550~nm), $\ell \sim 135{,}000$ (moderate angular complexity). Red bar: near-IR (1~$\mu$m), $\ell \sim 100{,}000$ (lower angular complexity). Brown bar: far-IR (10~$\mu$m), $\ell \sim 30{,}000$ (lowest angular complexity). 
\textbf{(D)} Virtual super-resolution chain showing $\delta x_{\text{virtual}} = \delta x_{\text{phys}} / \prod \gamma_k$. Green line with circles: resolution vs. catalyst stage. Starting point (physical observation): 21.14~mm (diffraction-limited). Catalyst 1 (texture prior): 7.05~mm (factor 3$\times$ improvement). Catalyst 2 (conservation laws): 2.35~mm (factor 3$\times$ improvement). Catalyst 3 (phase-lock network): 0.78~mm (factor 3$\times$ improvement). Final virtual resolution: 0.78~mm (factor 27$\times$ total improvement, shown in yellow box). Horizontal axis: catalyst stage (physical observation $\to$ catalyst 1 $\to$ catalyst 2 $\to$ catalyst 3 $\to$ virtual image). Vertical axis: resolution (logarithmic scale, $10^0$--$10^1$~mm). 
\textbf{(E)} Simulated observations of Apollo flag at three resolutions. Three grayscale images showing progressive detail. Left panel: single telescope ($\sim$21~m resolution), uniform gray field with noise, no flag visible. Center panel: interferometer ($\sim$0.021~m resolution), flag visible as small bright feature (yellow star), pole structure faintly visible. Right panel: virtual imaging ($\sim$0.8~mm resolution), flag clearly resolved with structural detail—vertical pole, horizontal crossbar, fabric texture visible (yellow star marks flag location). }
\label{fig:section8_validation}
\end{figure}

\subsection{Spectroscopic Partition Decomposition}

Different wavelengths probe different partition coordinates, enabling spectroscopic decomposition of partition structure.

\begin{theorem}[Spectral Partition Mapping]
\label{thm:spectral_partition}
Wavelength $\lambda$ maps to angular momentum partition coordinate $\ell$ via:
\begin{equation}
\ell(\lambda) \sim \frac{\lambda_{\text{ref}}}{\lambda}
\end{equation}
where $\lambda_{\text{ref}}$ is a reference scale (typically atomic size $\sim 1$ Å $= 10^{-10}$ m).
\end{theorem}

\begin{proof}
Oscillatory frequency $\omega = 2\pi c/\lambda$ determines partition angular complexity. From Theorem~\ref{thm:frequency_depth}:
\begin{equation}
\omega_\ell = \ell^2 \omega_0
\end{equation}

where $\omega_0$ is the fundamental frequency. Inverting:
\begin{equation}
\ell = \sqrt{\frac{\omega}{\omega_0}} = \sqrt{\frac{2\pi c/\lambda}{2\pi c/\lambda_0}} = \sqrt{\frac{\lambda_0}{\lambda}}
\end{equation}

For $\lambda_0 \sim 1$ Å (atomic scale):
\begin{equation}
\ell \sim \frac{1}{\sqrt{\lambda}}
\end{equation}

Alternatively, using energy $E = hc/\lambda$ and partition depth scaling $\ell \sim \sqrt{E/E_0}$:
\begin{equation}
\ell \sim \sqrt{\frac{hc/\lambda}{hc/\lambda_0}} = \sqrt{\frac{\lambda_0}{\lambda}}
\end{equation}

Different wavelengths probe different partition coordinates:
\begin{itemize}
    \item Short wavelength (UV, X-ray): High $\ell$ → fine angular structure, surface roughness
    \item Medium wavelength (visible): Mid $\ell$ → albedo, composition
    \item Long wavelength (IR, microwave): Low $\ell$ → thermal emission, subsurface structure
\end{itemize}

Spectroscopic observation accesses a range of $\ell$ coordinates, enabling partition decomposition.
\end{proof}

\begin{example}[Multi-Wavelength Lunar Imaging]
\label{ex:multiwavelength_lunar}
Observing the Moon at multiple wavelengths extracts complementary partition information:

\begin{itemize}
    \item \textbf{UV} ($\lambda \sim 300$ nm): 
        \begin{equation}
        \ell_{\text{UV}} \sim \sqrt{\frac{10^{-10}}{3 \times 10^{-7}}} \sim 0.018
        \end{equation}
        Probes surface roughness, grain boundaries, space weathering effects
    
    \item \textbf{Visible} ($\lambda \sim 550$ nm):
        \begin{equation}
        \ell_{\text{vis}} \sim \sqrt{\frac{10^{-10}}{5.5 \times 10^{-7}}} \sim 0.013
        \end{equation}
        Probes albedo, mineralogical composition (Fe, Ti content)
    
    \item \textbf{Near-IR} ($\lambda \sim 1000$ nm):
        \begin{equation}
        \ell_{\text{NIR}} \sim \sqrt{\frac{10^{-10}}{10^{-6}}} \sim 0.010
        \end{equation}
        Probes subsurface thermal emission, regolith depth
    
    \item \textbf{Thermal IR} ($\lambda \sim 10$ μm):
        \begin{equation}
        \ell_{\text{TIR}} \sim \sqrt{\frac{10^{-10}}{10^{-5}}} \sim 0.003
        \end{equation}
        Probes temperature distribution, rock vs. regolith thermal inertia
    
    \item \textbf{Radar} ($\lambda \sim 1$ cm--$1$ m):
        \begin{equation}
        \ell_{\text{radar}} \sim \sqrt{\frac{10^{-10}}{10^{-2}}} \sim 10^{-4}
        \end{equation}
        Penetrates regolith (depth $\sim$ meters), reveals subsurface rock layers
\end{itemize}

Combining all wavelengths via categorical morphisms yields composite partition signature with enhanced information content:
\begin{equation}
\Sigma_{\text{composite}} = \bigoplus_{\lambda} \Sigma_{\lambda}
\end{equation}

This is the categorical direct sum over wavelength channels. Each wavelength contributes orthogonal partition information, increasing total partition depth:
\begin{equation}
n_{\text{composite}} = \sqrt{\sum_{\lambda} n_{\lambda}^2}
\end{equation}

For $N_{\lambda}$ wavelength channels with comparable partition depths $n_{\lambda} \sim n$:
\begin{equation}
n_{\text{composite}} \sim n\sqrt{N_{\lambda}}
\end{equation}

Multispectral imaging with $N_{\lambda} = 10$ channels provides $\sqrt{10} \approx 3.2\times$ partition depth enhancement.
\end{example}

\begin{remark}
This establishes that spectroscopy is not merely "measuring color" but accessing different partition coordinates. Hyperspectral imaging (hundreds of wavelength channels) provides dense sampling of the $\ell$ coordinate, enabling detailed partition decomposition. This is the basis for mineralogical mapping of the Moon (Clementine, Chandrayaan-1, Lunar Reconnaissance Orbiter).
\end{remark}

\subsection{Temporal Coherence and Partition Lag}

Interferometry requires temporal coherence—the ability to maintain phase relationships over observation time. This relates to partition lag.

\begin{theorem}[Coherence from Partition Lag]
\label{thm:coherence_lag}
Interferometric fringe visibility requires partition lag $\tau_{\text{lag}}$ (Definition~\ref{def:partition_lag}) to be less than the observation time per fringe:
\begin{equation}
\tau_{\text{lag}} < \frac{T_{\text{obs}}}{N_{\text{fringes}}}
\end{equation}
where $N_{\text{fringes}}$ is the number of resolvable fringes and $T_{\text{obs}}$ is the total observation time.
\end{theorem}

\begin{proof}
Interferometric fringes arise from phase differences between light paths from different apertures. For baseline $B$ and wavelength $\lambda$, the number of fringes across the field of view is:
\begin{equation}
N_{\text{fringes}} \sim \frac{B}{\lambda}
\end{equation}

Each fringe corresponds to a phase change of $2\pi$. To resolve fringes, the partition state must remain determinate (phase coherent) over the time to measure one fringe:
\begin{equation}
\tau_{\text{fringe}} = \frac{T_{\text{obs}}}{N_{\text{fringes}}}
\end{equation}

If partition lag $\tau_{\text{lag}} > \tau_{\text{fringe}}$, the phase becomes indeterminate before the fringe is measured, destroying fringe visibility.

For optical frequencies ($\lambda \sim 550$ nm, $\omega = 2\pi c/\lambda \sim 3.4 \times 10^{15}$ rad/s), the partition lag:
\begin{equation}
\tau_{\text{lag}} = \frac{2\pi}{\omega} \sim \frac{1}{3.4 \times 10^{15}} \sim 3 \times 10^{-16} \text{ s}
\end{equation}

For lunar observation with $B = 10$ km:
\begin{equation}
N_{\text{fringes}} \sim \frac{10^4}{5.5 \times 10^{-7}} \sim 1.8 \times 10^{10}
\end{equation}

With observation time $T_{\text{obs}} \sim 1$ s:
\begin{equation}
\tau_{\text{fringe}} = \frac{1}{1.8 \times 10^{10}} \sim 5.6 \times 10^{-11} \text{ s}
\end{equation}

Since $\tau_{\text{lag}} \sim 3 \times 10^{-16}$ s $\ll \tau_{\text{fringe}} \sim 5.6 \times 10^{-11}$ s, coherence is easily maintained at the partition level.

The practical limitation is not partition lag but:
\begin{itemize}
    \item \textbf{Atmospheric turbulence}: $\tau_{\text{turb}} \sim 10$ ms (ground-based)
    \item \textbf{Mechanical vibration}: $\tau_{\text{vib}} \sim 1$ ms (baseline instability)
    \item \textbf{Thermal drift}: $\tau_{\text{thermal}} \sim 1$ s (temperature changes)
\end{itemize}

These are addressed via:
\begin{itemize}
    \item Adaptive optics (corrects atmospheric turbulence)
    \item Vibration isolation (stabilizes baseline)
    \item Space-based interferometry (eliminates atmosphere)
\end{itemize}
\end{proof}

\begin{remark}
This establishes that partition lag is not a fundamental limitation for optical interferometry. The coherence time is limited by environmental factors (atmosphere, vibration), not by intrinsic partition completion dynamics. Virtual interferometry (computational apertures) eliminates environmental limitations by operating in categorical space rather than physical space.
\end{remark}

\subsection{Summary: Interferometric Resolution Enhancement}

This section establishes that:

\begin{enumerate}
    \item \textbf{Telescopes have partition depth} $n \sim D/\lambda$ (Theorem~\ref{thm:single_aperture_depth}): Resolution emerges from partition geometry, not wave diffraction
    
    \item \textbf{Interferometry enhances partition depth} to $n_{\text{eff}} \sim B/\lambda$ (Theorem~\ref{thm:interferometric_depth}): Baseline $B$ replaces aperture $D$
    
    \item \textbf{10 km baseline achieves 2.1 cm resolution} (Example~\ref{ex:lunar_interferometry}): Sufficient to resolve Apollo landers and flags
    
    \item \textbf{Virtual resolution reaches 0.78 mm} (Corollary~\ref{cor:apollo_virtual}): Information catalysis provides $27\times$ enhancement, resolving bootprint treads
    
    \item \textbf{Spectroscopy accesses different partition coordinates} (Theorem~\ref{thm:spectral_partition}): Multispectral imaging provides $\sqrt{N_{\lambda}}$ enhancement
    
    \item \textbf{Partition lag is not limiting} (Theorem~\ref{thm:coherence_lag}): Coherence maintained at optical frequencies; environmental factors dominate
\end{enumerate}

These results establish that cm-scale and mm-scale lunar surface resolution from Earth is achievable through:
\begin{itemize}
    \item Physical interferometry: $B \sim 10$ km → $\delta x \sim 2$ cm
    \item Virtual interferometry: Information catalysis → $\delta x \sim 0.8$ mm
    \item Multispectral imaging: $N_{\lambda} \sim 10$ channels → $3\times$ enhancement
\end{itemize}

Combined, these techniques enable resolution exceeding orbital imaging (LRO: $\sim 0.5$ m) from Earth distance (384,400 km), achieved through categorical measurement rather than photonic imaging.
