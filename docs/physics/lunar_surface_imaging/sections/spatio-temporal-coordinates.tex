\section{Spatio-Temporal Coordinates from Partition Geometry}
\label{sec:spatiotemporal}

\subsection{Time as Categorical Completion Order}

Time is not an independent parameter existing prior to physical processes—it emerges from the ordering of partition completion events.

\begin{definition}[Temporal Coordinate]
\label{def:temporal_coordinate}
The temporal coordinate $t$ is the ordering parameter for categorical state completion:
\begin{equation}
t: \{\text{Partition configurations}\} \to \mathbb{R}^+
\end{equation}
such that later times correspond to more completed (determinate) categorical states. The function $t$ assigns a real number to each partition configuration, with $t_A < t_B$ meaning configuration $\Sigma_A$ completes before configuration $\Sigma_B$.
\end{definition}

\begin{remark}
This differs fundamentally from the Newtonian conception of time as an absolute, pre-existing parameter. Here, time is relational: it measures the ordering of partition events, not the "flow" of an independent temporal dimension. There is no time "between" partition events—time is the discrete sequence of partition completions, with continuous time emerging in the limit of dense partition sequences.
\end{remark}

\begin{theorem}[Arrow of Time from Partition Entropy]
\label{thm:arrow_of_time}
Temporal direction is determined by partition entropy increase:
\begin{equation}
\frac{dS_{\text{part}}}{dt} = k_B \frac{d(M \ln n)}{dt} \geq 0
\end{equation}
Each partition operation generates entropy $\Delta S = k_B M \ln n$, making time irreversible at the categorical level.
\end{theorem}

\begin{proof}
Partition operations create categorical boundaries, transforming undetermined residue into determinate states. The process is thermodynamically irreversible: once boundaries crystallize, the information about which specific continuous value was "chosen" is lost.

\textbf{Before partition:} Continuous domain with infinite distinguishable states (in principle). Entropy:
\begin{equation}
S_{\text{before}} = k_B \ln(\infty) \to \infty \quad \text{(formal, ill-defined)}
\end{equation}

\textbf{After partition:} Discrete domain with $n^M$ distinguishable states. Entropy:
\begin{equation}
S_{\text{after}} = k_B \ln(n^M) = k_B M \ln n \quad \text{(finite, well-defined)}
\end{equation}

The act of partitioning generates entropy through:
\begin{enumerate}
    \item \textbf{Discretization}: Continuous $\to$ discrete (information loss about exact continuous values)
    \item \textbf{Boundary crystallization}: Undetermined $\to$ determined (loss of superposition)
    \item \textbf{Distinguishability creation}: Indistinguishable $\to$ distinguishable (creation of categorical structure)
\end{enumerate}

Each partition step increases $M$ (number of dimensions partitioned) or $n$ (partition depth), increasing entropy:
\begin{equation}
\Delta S = k_B \Delta(M \ln n) > 0
\end{equation}

This establishes the thermodynamic arrow of time: partition completion proceeds in the direction of increasing categorical entropy. Time reversal would require "unpartitioning"—erasing categorical boundaries and restoring undetermined residue—which is thermodynamically forbidden.
\end{proof}

\begin{corollary}[Irreversibility of Observation]
\label{cor:irreversibility_observation}
Measurement (accessing partition signatures) is irreversible because it completes partition boundaries. The act of observation creates temporal ordering.
\end{corollary}

\begin{proof}
Observation accesses partition signatures $(n, \ell, m, s)$ (Corollary~\ref{cor:measurement}). Accessing a signature requires the partition boundary to be determinate—the system must be in a definite categorical state, not undetermined residue.

The measurement process completes partitions:
\begin{equation}
\text{Undetermined residue} \xrightarrow{\text{measurement}} \text{Determinate partition state}
\end{equation}

This is irreversible (Theorem~\ref{thm:arrow_of_time}), establishing that observation creates temporal ordering. The "collapse" of quantum superposition is the crystallization of partition boundaries through measurement-induced completion.
\end{proof}

\subsection{Space-Time as Unified Partition Structure}

Space and time are not separate entities—they are complementary aspects of partition geometry.

\begin{theorem}[Space-Time Unification via Partitions]
\label{thm:spacetime_partition}
Four-dimensional space-time $(x, y, z, t)$ emerges from partition coordinates and completion ordering:
\begin{equation}
\begin{pmatrix} x \\ y \\ z \\ t \end{pmatrix} \leftrightarrow \begin{pmatrix} n, \ell, m \\ \text{completion order} \end{pmatrix}
\end{equation}
Spatial coordinates $(x,y,z)$ arise from partition structure $(n,\ell,m)$ (Theorem~\ref{thm:spatial_emergence}); temporal coordinate $t$ arises from categorical completion sequence (Definition~\ref{def:temporal_coordinate}).
\end{theorem}

\begin{proof}
From Theorem~\ref{thm:spatial_emergence}, spatial coordinates emerge from angular partition structure:
\begin{align}
r &\sim n a_0 \quad \text{(radial from partition depth)} \\
(\theta, \phi) &\sim (\ell, m) \quad \text{(angular from node arrangement)}
\end{align}

Temporal coordinate emerges from completion order (Theorem~\ref{thm:arrow_of_time}):
\begin{equation}
t \sim \text{sequence number of partition completion}
\end{equation}

Both are aspects of the same partition process:
\begin{itemize}
    \item \textbf{Spatial structure}: Which partition configurations exist simultaneously (parallel partitioning across different spatial regions)
    \item \textbf{Temporal structure}: In what sequence do partition states become determinate (sequential completion within a given region)
\end{itemize}

The space-time interval combines spatial partition differences and temporal completion differences:
\begin{equation}
ds^2 = -c^2 dt^2 + dx^2 + dy^2 + dz^2
\end{equation}
where $c$ is the maximum partition completion propagation speed—the speed at which partition boundaries can crystallize across space.

The minus sign in the temporal term reflects the complementarity of space and time: increasing spatial separation (large $dx^2 + dy^2 + dz^2$) allows simultaneous partition completion (small $dt^2$), while decreasing spatial separation (small $dx^2 + dy^2 + dz^2$) requires sequential completion (large $dt^2$).
\end{proof}

\begin{corollary}[Speed of Light as Partition Propagation Limit]
\label{cor:speed_of_light}
The speed of light $c$ is the maximum speed at which partition boundaries can propagate:
\begin{equation}
c = \lim_{\Delta t \to 0} \frac{\Delta x}{\Delta t} = \text{maximum partition completion propagation speed}
\end{equation}
\end{corollary}

\begin{proof}
Partition completion at one location influences partition completion at nearby locations through phase-lock network coupling (Definition~\ref{def:phase_lock_network}). The coupling propagates at finite speed determined by the oscillatory frequency and wavelength:
\begin{equation}
c = \lambda \omega = \frac{2\pi}{\omega} \cdot \omega = \frac{\omega}{k}
\end{equation}

For electromagnetic oscillations (photons), this is the speed of light. For other partition completion mechanisms (acoustic, thermal), the propagation speed differs. The speed of light is the maximum because electromagnetic coupling is the fastest partition completion mechanism in vacuum.

No partition boundary can crystallize faster than $c$ because that would require instantaneous action at a distance—violating the sequential nature of partition completion.
\end{proof}

\begin{remark}
This establishes that special relativity emerges from partition geometry. The Lorentz transformation, time dilation, and length contraction are consequences of the unified space-time partition structure, not independent postulates. The constancy of the speed of light reflects the invariance of partition completion propagation speed across different reference frames.
\end{remark}

\subsection{Gravitational Phase-Lock Networks}

Gravitational interaction emerges from large-scale phase-lock networks coupling massive bodies.

\begin{theorem}[Gravitational Coupling from Partition Networks]
\label{thm:gravitational_coupling}
For massive bodies with partition depths $n_1, n_2$ separated by distance $r$, phase-lock coupling gives gravitational potential:
\begin{equation}
V_{\text{grav}}(r) = -\frac{G M_1 M_2}{r}
\end{equation}
where mass relates to partition capacity by:
\begin{equation}
M = 2n^2 m_p \cdot N_{\text{shells}}
\end{equation}
with $m_p$ the proton mass and $N_{\text{shells}}$ the number of filled partition shells.
\end{theorem}

\begin{proof}
Large partition configurations (massive bodies) create extensive phase-lock networks. Each particle in body 1 couples to each particle in body 2 through gravitational phase-lock interaction.

For body 1 with $N_1$ particles and body 2 with $N_2$ particles, the total coupling:
\begin{equation}
V_{\text{total}} = -\sum_{i=1}^{N_1}\sum_{j=1}^{N_2} \frac{G m_i m_j}{r_{ij}}
\end{equation}

For spherically symmetric mass distributions separated by $r \gg R_1, R_2$ (where $R_1, R_2$ are the body radii), the shell theorem applies:
\begin{equation}
V_{\text{grav}} \approx -\frac{G M_1 M_2}{r}
\end{equation}
where $M_1 = \sum_{i=1}^{N_1} m_i$ and $M_2 = \sum_{j=1}^{N_2} m_j$ are the total masses.

The relationship between mass and partition depth:
\begin{equation}
M = N_{\text{particles}} \cdot m_p \approx 2n^2 m_p \cdot N_{\text{shells}}
\end{equation}

For the Moon with $M_{\text{Moon}} \sim 7.3 \times 10^{22}$ kg:
\begin{equation}
N_{\text{particles}} = \frac{M_{\text{Moon}}}{m_p} \sim \frac{7.3 \times 10^{22} \text{ kg}}{1.67 \times 10^{-27} \text{ kg}} \sim 4.4 \times 10^{49}
\end{equation}

If organized into shells with capacity $2n^2$ per shell:
\begin{equation}
N_{\text{shells}} \sim \frac{4.4 \times 10^{49}}{2n^2}
\end{equation}

For $n \sim 10^{10}$ (typical atomic-scale partition depth):
\begin{equation}
N_{\text{shells}} \sim \frac{4.4 \times 10^{49}}{2 \times 10^{20}} \sim 2.2 \times 10^{29}
\end{equation}

This establishes that gravitational coupling is the long-range tail of phase-lock network interactions. The $r^{-1}$ dependence (compared to Van der Waals $\sim r^{-6}$, dipole $\sim r^{-3}$) reflects the monopole nature of mass—all particles contribute additively to gravitational coupling, whereas higher multipole moments (dipole, quadrupole) cancel for symmetric distributions.
\end{proof}

\begin{corollary}[Gravitational Constant from Partition Parameters]
\label{cor:gravitational_constant}
The gravitational constant $G$ is determined by partition completion dynamics and phase-lock coupling strength:
\begin{equation}
G \sim \frac{\alpha_{\text{grav}} \hbar c}{m_p^2}
\end{equation}
where $\alpha_{\text{grav}}$ is the dimensionless gravitational coupling strength.
\end{corollary}

\begin{proof}
Phase-lock coupling strength (Definition~\ref{def:phase_lock_network}) has form:
\begin{equation}
V_{ij} = -\alpha_{ij} \cos(\phi_i - \phi_j) \cdot f(r_{ij})
\end{equation}

For gravitational coupling, $f(r) = 1/r$ and $\alpha_{\text{grav}}$ set the overall strength. Dimensional analysis requires:
\begin{equation}
[G] = \frac{[\text{energy}] \cdot [\text{length}]}{[\text{mass}]^2} = \frac{\text{J} \cdot \text{m}}{\text{kg}^2}
\end{equation}

Constructing from fundamental constants:
\begin{equation}
G \sim \frac{\hbar c}{m_p^2} \cdot \alpha_{\text{grav}}
\end{equation}

The measured value $G \approx 6.67 \times 10^{-11}$ m$^3$ kg$^{-1}$ s$^{-2}$ implies:
\begin{equation}
\alpha_{\text{grav}} \sim \frac{G m_p^2}{\hbar c} \sim 10^{-38}
\end{equation}

This extremely small coupling constant reflects the weakness of gravitational interaction compared to electromagnetic ($\alpha_{\text{EM}} \sim 1/137$) or strong ($\alpha_{\text{strong}} \sim 1$) interactions. Gravity is weak because it couples to mass (partition capacity) rather than charge or colour.
\end{proof}

\begin{remark}
This derivation establishes that gravity is not a fundamental force but an emergent phenomenon from phase-lock network coupling at large scales. The weakness of gravity ($\alpha_{\text{grav}} \sim 10^{-38}$) compared to other interactions reflects its origin as a collective effect of many particles rather than a direct coupling between individual particles.
\end{remark}


\begin{figure}[htbp]
\centering
\includegraphics[width=\textwidth]{figures/section_5_validation.png}
\caption{\textbf{Section 5 validation: Spatio-temporal coordinates showing space-time emergence from partition geometry with gravitational coupling and hierarchical structure.} 
\textbf{(A) Time from partition order} $dS/dt > 0$ (arrow of time) showing cumulative entropy vs. temporal coordinate (completion order). Horizontal axis: temporal coordinate $t$ (completion order, 0 to 8 arbitrary units). Vertical axis: cumulative entropy $S/k_B$ (0 to 14). Blue curve with data points: monotonically increasing entropy as partitions complete sequentially. 
\textbf{(B) Space-time unification} $(x, y, z, t)$ from $(n, \ell, m, s, \text{order})$ showing spatial orbit in 2D projection. Horizontal axis: spatial $X$ (from $n, \ell, m$, range $-2$ to $+2$ arbitrary units). Vertical axis: spatial $Y$ (from $n, \ell, m$, range $-2$ to $+2$ arbitrary units). Colored dots: sequential positions along orbit, color-coded by temporal coordinate (blue = early, yellow = late, scale 0 to 10). Circular trajectory shows periodic motion. 
\textbf{(C) Gravitational phase-lock} $V = -G(2n_1^2 m_p)(2n_2^2 m_p)/r$ showing gravitational coupling strength vs. partition depth. Horizontal axis: partition depth $n$ (log scale, $10^{10}$ to $10^{15}$). Vertical axis: gravitational coupling $|V|$ (Joules, log scale, $10^{-34}$ to $10^{-18}$). Red line: linear relationship on log-log plot, indicating power-law scaling $V \sim n^4$ (since mass $M \sim n^2$, and $V \sim M^2/r$).
\textbf{(D) Earth-Moon system barycentric coordinates} showing orbital configuration in 2D projection. Horizontal axis: $X$ (1000 km, 0 to 400). Vertical axis: $Y$ (1000 km, $-400$ to $+400$). Red star at origin: Earth. Gray dots: Moon positions along orbit (scaled for visibility). Blue dot: barycenter (center of mass). Circular trajectory (gray dashed line): Moon's orbit around barycenter. 
\textbf{(E) Hierarchical structure} $n_{\text{Sun}} \gg n_{\text{Earth}} > n_{\text{Moon}} \gg n_{\text{atom}}$ showing effective partition depth vs. mass. Horizontal axis: effective partition depth $n_{\text{eff}}$ (log scale, $10^0$ to $10^{32}$). Vertical axis: mass (kg, log scale, $10^{-27}$ to $10^{28}$).}
\label{fig:section5_validation}
\end{figure}



\subsection{Coordinate Systems for Astronomical Bodies}

The Earth-Moon system defines a natural hierarchical coordinate system based on partition structure.

\begin{definition}[Barycentric Coordinates]
\label{def:barycentric}
The Earth-Moon barycenter (centre of mass) defines the origin for the two-body system:
\begin{equation}
\mathbf{R}_{\text{barycenter}} = \frac{M_{\text{Earth}} \mathbf{R}_{\text{Earth}} + M_{\text{Moon}} \mathbf{R}_{\text{Moon}}}{M_{\text{Earth}} + M_{\text{Moon}}}
\end{equation}
where $\mathbf{R}_{\text{Earth}}$ and $\mathbf{R}_{\text{Moon}}$ are position vectors relative to an external reference frame (e.g., Sun-centred).
\end{definition}

\begin{theorem}[Hierarchical Partition Structure]
\label{thm:hierarchical_partition}
Astronomical systems are organised by partition depth hierarchy:
\begin{equation}
n_{\text{Sun}} \gg n_{\text{Earth-Moon}} \gg n_{\text{Earth}}, n_{\text{Moon}} \gg n_{\text{individual atoms}}
\end{equation}
Lower-depth subsystems move within phase-locked networks of higher-depth systems.
\end{theorem}

\begin{proof}
Partition depth scales with mass (Theorem~\ref{thm:gravitational_coupling}):
\begin{equation}
n \sim \sqrt{\frac{M}{2m_p \cdot N_{\text{shells}}}}
\end{equation}

For astronomical bodies:
\begin{align}
M_{\text{Sun}} &\sim 2 \times 10^{30} \text{ kg} \quad \Rightarrow \quad n_{\text{Sun}} \sim 10^{40} \\
M_{\text{Earth}} &\sim 6 \times 10^{24} \text{ kg} \quad \Rightarrow \quad n_{\text{Earth}} \sim 10^{37} \\
M_{\text{Moon}} &\sim 7 \times 10^{22} \text{ kg} \quad \Rightarrow \quad n_{\text{Moon}} \sim 10^{36}
\end{align}

(These are total partition depths accounting for all particles. Surface partition depths relevant for observation are lower, $n_{\text{surface}} \sim 10^{30}$ as established in Theorem~\ref{thm:depth_hierarchy}.)

The hierarchy establishes that:
\begin{itemize}
    \item Earth and Moon orbit within the Sun's phase-lock network (Solar System)
    \item Moon orbits within Earth's phase-lock network (Earth-Moon system)
    \item Surface features exist within the Moon's internal phase-lock network
\end{itemize}

Each level of the hierarchy defines a natural coordinate system:
\begin{enumerate}
    \item \textbf{Heliocentric}: Sun at origin, planets orbit
    \item \textbf{Geocentric}: Earth at origin, Moon orbits
    \item \textbf{Selenocentric}: Moon at origin, surface features fixed
\end{enumerate}

Position is relational partition structure, not intrinsic property. The Moon's position is defined relative to Earth (and Sun), not as an absolute coordinate in pre-existing space.
\end{proof}

\begin{corollary}[Orbital Dynamics from Phase-Lock Equilibrium]
\label{cor:orbital_dynamics}
The Moon's orbit around Earth is the equilibrium configuration of the Earth-Moon phase-lock network, minimizing total potential energy:
\begin{equation}
\frac{d}{dt}\left( T + V_{\text{grav}} \right) = 0
\end{equation}
where $T$ is kinetic energy and $V_{\text{grav}}$ is gravitational potential (Theorem~\ref{thm:gravitational_coupling}).
\end{corollary}

\begin{proof}
The Earth-Moon system seeks minimum energy configuration. For two bodies with masses $M_1, M_2$ separated by distance $r$:
\begin{equation}
E_{\text{total}} = \frac{1}{2}\mu v^2 - \frac{G M_1 M_2}{r}
\end{equation}
where $\mu = M_1 M_2/(M_1 + M_2)$ is reduced mass and $v$ is relative velocity.

Equilibrium (circular orbit) requires:
\begin{equation}
\frac{\mu v^2}{r} = \frac{G M_1 M_2}{r^2} \quad \Rightarrow \quad v = \sqrt{\frac{G(M_1 + M_2)}{r}}
\end{equation}

For Earth-Moon system with $r \sim 3.84 \times 10^8$ m:
\begin{equation}
v_{\text{Moon}} = \sqrt{\frac{G M_{\text{Earth}}}{r}} \sim 1.0 \text{ km/s}
\end{equation}

(Measured orbital velocity: $v_{\text{Moon}} \approx 1.022$ km/s, confirming the phase-lock equilibrium.)

The orbit is not a trajectory through pre-existing space but the equilibrium configuration of partition structure minimizing phase-lock network energy. Orbital mechanics emerges from partition dynamics, not from separate gravitational "force" acting on bodies moving through space.
\end{proof}

\begin{remark}
This establishes that celestial mechanics (Kepler's laws, orbital dynamics) emerges from partition structure equilibrium. The Moon orbits Earth not because of a "force" pulling it but because the orbital configuration is the minimum-energy state of the Earth-Moon phase-lock network. This is a subtle but important distinction: forces are effective descriptions of partition network gradients, not fundamental entities.
\end{remark}

\subsection{Summary: Space-Time and Gravitational Structure}

This section establishes that:

\begin{enumerate}
    \item \textbf{Time emerges from partition completion order} (Theorem~\ref{thm:arrow_of_time}): Temporal direction is determined by entropy increase through partition boundary crystallization. Time is not pre-existing but relational.
    
    \item \textbf{Space-time is unified partition structure} (Theorem~\ref{thm:spacetime_partition}): Four-dimensional space-time emerges from partition coordinates $(n, \ell, m)$ and completion ordering. Special relativity follows from partition propagation limits.
    
    \item \textbf{Gravity emerges from phase-lock networks} (Theorem~\ref{thm:gravitational_coupling}): Gravitational interaction is the long-range tail of phase-lock coupling, with strength determined by partition capacity (mass). Newton's law of gravitation is recovered.
    
    \item \textbf{Astronomical systems organize hierarchically} (Theorem~\ref{thm:hierarchical_partition}): Sun → Earth-Moon → individual bodies form nested partition structures. Orbital dynamics emerges from phase-lock equilibrium.
\end{enumerate}

These results establish that the Moon exists as a massive body ($M \sim 7 \times 10^{22}$ kg, $n \sim 10^{36}$) in gravitational phase-lock equilibrium with Earth, orbiting at distance $r \sim 3.84 \times 10^8$ m with velocity $v \sim 1.0$ km/s. The Moon's position and motion are not trajectories through pre-existing space-time but equilibrium configurations of partition structure.
