\section{Forced Localization Dynamics}

\subsection{Hamiltonian with Perturbations}

The total Hamiltonian including both perturbations is:
\begin{equation}
\hat{H}(t) = \hat{H}_0 + \hat{V}_{\mathcal{P}_1}(t) + \hat{V}_{\mathcal{P}_2}(t)
\end{equation}

where $\hat{H}_0 = \frac{\hat{p}^2}{2m} + V_0(\mathbf{r})$ is the unperturbed Hamiltonian (kinetic plus Coulomb potential), and $\hat{V}_{\mathcal{P}_i}$ are the perturbation potentials.

\subsection{Eigenstate Structure Under Strong Perturbation}

When $|V_{\mathcal{P}}| \gg |V_0|$ in the active region, the eigenstates of $\hat{H}$ are approximately position eigenstates localized in the active region.

\subsubsection{Perturbative Limit}

In the weak perturbation regime ($|V_{\mathcal{P}}| \ll |V_0|$), the eigenstates are approximately the unperturbed eigenstates $|\psi_n\rangle$ of $\hat{H}_0$, with small corrections:
\begin{equation}
|\psi_n'\rangle = |\psi_n\rangle + \sum_{m \neq n} \frac{\langle \psi_m | \hat{V}_{\mathcal{P}} | \psi_n \rangle}{E_n - E_m} |\psi_m\rangle + O(V_{\mathcal{P}}^2)
\end{equation}

This regime is not useful for localization because the state remains delocalized.

\subsubsection{Strong Perturbation Regime}

In the strong perturbation regime ($|V_{\mathcal{P}}| \gg |V_0|$), we invert the roles: treat $\hat{H}_0$ as a perturbation to $\hat{V}_{\mathcal{P}}$. The zeroth-order eigenstates are eigenstates of $\hat{V}_{\mathcal{P}}$.

For the step-function perturbation $V_{\mathcal{P}_1}(x, z) = e E_0 f(x) z$, the eigenstates in the active region $(x < L/3)$ are plane waves in $z$:
\begin{equation}
\psi_{k_z}(z) = \frac{1}{\sqrt{L_z}} e^{i k_z z}
\end{equation}
with energy $E_{k_z} = \frac{\hbar^2 k_z^2}{2m} + e E_0 z_0$, where $z_0$ is the mean $z$-position.

The key point is that these eigenstates have definite position in $x$ (localized to $x < L/3$ by the boundary condition $f(x) = 0$ for $x > L/3$) while remaining delocalized in $z$. This provides partial localization: we know $x$ but not $z$.

\subsection{Adiabatic vs Sudden Perturbations}

The transition between eigenstates depends on how fast the perturbation is turned on relative to the system's natural timescales.

\subsubsection{Adiabatic Theorem}

If the perturbation varies slowly compared to the energy level spacings:
\begin{equation}
\left| \frac{d\hat{H}/dt}{(E_n - E_m)^2} \langle \psi_m | \partial_t \psi_n \rangle \right| \ll 1
\end{equation}
then the system remains in the instantaneous eigenstate of $\hat{H}(t)$. If initially in $|\psi_n(0)\rangle$, it evolves to $|\psi_n(t)\rangle$ (the eigenstate that adiabatically connects to the initial state).

For our hydrogen system with level spacing $\Delta E \sim 10$ eV and perturbation rise time $\tau_{\text{rise}} \sim 10^{-8}$ s:
\begin{equation}
\frac{1}{\tau_{\text{rise}}} \sim 10^8 \text{ s}^{-1} \ll \frac{\Delta E}{\hbar} \sim \frac{10 \text{ eV}}{10^{-15} \text{ eV·s}} \sim 10^{16} \text{ s}^{-1}
\end{equation}

The perturbation is adiabatic, so the electron follows the instantaneous eigenstate. If initially in the 1s ground state (delocalized), it adiabatically transitions to the ground state of $\hat{H} = \hat{H}_0 + \hat{V}_{\mathcal{P}}$ (localized to the active region if the particle is there, or to the inactive region if not).

\subsubsection{Sudden Perturbation}

If the perturbation is turned on instantaneously ($\tau_{\text{rise}} \to 0$), the wavefunction does not have time to adjust. The sudden approximation gives:
\begin{equation}
|\psi(t = 0^+)\rangle = |\psi(t = 0^-)\rangle
\end{equation}

The wavefunction is unchanged, but it is now a superposition of eigenstates of the new Hamiltonian $\hat{H} + \hat{V}_{\mathcal{P}}$. The system then evolves under the new Hamiltonian, exhibiting Rabi oscillations between the eigenstates.

For localization, the adiabatic regime is preferable: the system smoothly transitions to a localized eigenstate without oscillations.

\subsection{Response Dynamics and Measurement}

Once the perturbation is applied and the system has settled into an eigenstate (adiabatic) or superposition (sudden), we measure the response.

\subsubsection{Response Observable}

The "response" to perturbation $\mathcal{P}_1$ is measured by the observable:
\begin{equation}
\hat{R}_1 = \int_{\Omega_{\mathcal{P}_1}} d^3r \, |\mathbf{r}\rangle \langle \mathbf{r}|
\end{equation}

This is a projection operator onto the active region. The expectation value is:
\begin{equation}
\langle \hat{R}_1 \rangle = \int_{\Omega_{\mathcal{P}_1}} |\psi(\mathbf{r})|^2 d^3r
\end{equation}

This is the probability of finding the particle in the active region. If $\langle \hat{R}_1 \rangle > 0.5$ (threshold), we declare $r_1 = 1$ (response). Otherwise, $r_1 = 0$ (no response).

\subsubsection{Time Evolution}

After the perturbation is applied at $t = 0$, the wavefunction evolves as:
\begin{equation}
|\psi(t)\rangle = e^{-i\hat{H}t/\hbar} |\psi(0)\rangle
\end{equation}

For strong perturbation, the dominant term in $\hat{H}$ is $\hat{V}_{\mathcal{P}}$, so:
\begin{equation}
|\psi(t)\rangle \approx e^{-i\hat{V}_{\mathcal{P}}t/\hbar} |\psi(0)\rangle
\end{equation}

This causes phase accumulation proportional to the perturbation strength. The response $\langle \hat{R}_1(t) \rangle$ oscillates with frequency $\omega \sim V_{\mathcal{P}}/\hbar$. We measure at time $t_{\text{meas}}$ chosen such that the oscillation is at a maximum (for particles in the active region) or minimum (for particles outside).

\subsection{Measurement Backaction}

The act of measuring the response $\langle \hat{R}_1 \rangle$ projects the wavefunction onto either the active region (if $r_1 = 1$) or the inactive region (if $r_1 = 0$).

\subsubsection{Post-Measurement State}

If $r_1 = 1$, the post-measurement state is:
\begin{equation}
|\psi'\rangle = \frac{\hat{P}_{\Omega_{\mathcal{P}_1}} |\psi\rangle}{\sqrt{\langle \psi | \hat{P}_{\Omega_{\mathcal{P}_1}} | \psi \rangle}}
\end{equation}
where $\hat{P}_{\Omega_{\mathcal{P}_1}}$ is the projection operator onto $\Omega_{\mathcal{P}_1}$.

This is a position measurement localized to region $\Omega_{\mathcal{P}_1}$. By Heisenberg uncertainty, it introduces momentum uncertainty:
\begin{equation}
\Delta p \sim \frac{\hbar}{\Delta x} \sim \frac{\hbar}{L/3}
\end{equation}

For $L = 10 a_0$, this gives $\Delta p \sim 3\hbar/10a_0 \approx 3 p_{\text{Bohr}}$ (three times the Bohr momentum). This is not negligible.

\subsubsection{Categorical Measurement Reduces Backaction}

However, we are not measuring position directly but the categorical observable "which partition?" The partition size is $\Delta x \sim L/3$, so the position uncertainty is $\Delta x \sim L/3$, and the momentum disturbance is $\Delta p \sim 3\hbar/L$, which is three times smaller than measuring position to precision $L/9$ (which would be the precision after 3 trisection steps in binary search).

By measuring categorical coordinates instead of physical position, we reduce backaction by the trisection factor.

\subsection{Decoherence Considerations}

Interactions with the environment cause decoherence, destroying quantum superposition. For localization, decoherence is actually helpful: it suppresses superposition states and forces the particle into a definite position.

\subsubsection{Decoherence Rate}

The decoherence rate for a particle in a trap is:
\begin{equation}
\Gamma_{\text{dec}} \sim \frac{k_B T}{\hbar} \cdot \frac{1}{Q}
\end{equation}
where $Q$ is the quality factor of the trap. For $T = 4$ K and $Q = 10^6$:
\begin{equation}
\Gamma_{\text{dec}} \sim \frac{0.34 \text{ meV}}{10^{-15} \text{ eV·s}} \cdot 10^{-6} \sim 3 \times 10^5 \text{ s}^{-1}
\end{equation}

The decoherence time is $\tau_{\text{dec}} \sim 3 \times 10^{-6}$ s, much longer than the perturbation duration $\tau_{\mathcal{P}} \sim 10^{-8}$ s. Therefore, decoherence is negligible during a single trisection step.

\subsubsection{Decoherence-Assisted Localization}

Over many trisection steps ($N \sim 100$ for trajectory tracking), cumulative decoherence suppresses superposition, forcing the electron into a mixed state (classical probability distribution) rather than a pure state (quantum superposition). This is acceptable for localization: we only need to know which region the particle occupies, not its quantum phase.

\subsection{Rabi Oscillations Between Localized States}

If the perturbation is time-dependent, the system can undergo Rabi oscillations between different localized states.

\subsubsection{Two-Level System}

Consider a particle that can occupy two regions $A$ and $B$ with energies $E_A$ and $E_B$. A time-dependent perturbation $V(t) = V_0 \cos(\omega t)$ couples the two states with matrix element $V_{AB} = \langle A | V | B \rangle$.

The Rabi frequency is:
\begin{equation}
\Omega_R = \frac{V_{AB}}{\hbar}
\end{equation}

The population oscillates as:
\begin{equation}
P_A(t) = \cos^2(\Omega_R t / 2)
\end{equation}

If we measure at time $t = \pi/(2\Omega_R)$, the populations are equalized: $P_A = P_B = 1/2$. This is the worst case for response detection. To avoid this, we choose measurement time $t_{\text{meas}}$ such that $\Omega_R t_{\text{meas}} \ll 1$ (measure before oscillations develop) or $\Omega_R t_{\text{meas}} \approx 2\pi$ (measure at a full cycle).

For our system with $V_{AB} \sim 1$ eV, $\Omega_R \sim 10^{15}$ s$^{-1}$, and $t_{\text{meas}} \sim 10^{-8}$ s:
\begin{equation}
\Omega_R t_{\text{meas}} \sim 10^{15} \cdot 10^{-8} = 10^7 \gg 1
\end{equation}

Many Rabi cycles occur during measurement. To obtain a definite response, we time-average over multiple cycles or measure at a phase-locked instant.

\subsection{Threshold Behavior and Sensitivity}

The response detection has a threshold: perturbations below a critical strength $V_{\text{crit}}$ do not produce measurable response.

\subsubsection{Signal-to-Noise Ratio}

The signal (response) is $S \sim V_{\mathcal{P}}$, and the noise is $N \sim \sqrt{k_B T}$ (thermal). The signal-to-noise ratio is:
\begin{equation}
\text{SNR} = \frac{V_{\mathcal{P}}}{\sqrt{k_B T}}
\end{equation}

For reliable detection, $\text{SNR} > 10$. This requires:
\begin{equation}
V_{\mathcal{P}} > 10 \sqrt{k_B T} \approx 10 \sqrt{0.34 \text{ meV}} \approx 6 \text{ meV}
\end{equation}

Our perturbations ($V_{\mathcal{P}_1} \sim 1$ eV, $V_{\mathcal{P}_2} \sim 0.5$ meV) both exceed this threshold, though $V_{\mathcal{P}_2}$ is marginal. Increasing $V_{\mathcal{P}_2}$ (by increasing magnetic field) would improve SNR.

\subsubsection{False Positive/Negative Rates}

If the threshold is set too low, thermal fluctuations cause false positives (particle not in active region but noise mimics response). If too high, weak responses are missed (false negatives). The optimal threshold $V_{\text{th}}$ minimizes the total error rate:
\begin{equation}
p_{\text{error}} = p_{\text{fp}} + p_{\text{fn}}
\end{equation}

For Gaussian noise with variance $\sigma^2 = k_B T$, the error rates are:
\begin{align}
p_{\text{fp}} &= \int_{V_{\text{th}}}^\infty \frac{1}{\sqrt{2\pi\sigma^2}} e^{-(V - 0)^2/(2\sigma^2)} dV \\
p_{\text{fn}} &= \int_{-\infty}^{V_{\text{th}}} \frac{1}{\sqrt{2\pi\sigma^2}} e^{-(V - V_{\mathcal{P}})^2/(2\sigma^2)} dV
\end{align}

The optimal threshold is $V_{\text{th}} = V_{\mathcal{P}}/2$ (midpoint), giving $p_{\text{error}} \approx 2 \Phi(-V_{\mathcal{P}}/(2\sigma))$, where $\Phi$ is the cumulative normal distribution. For $V_{\mathcal{P}}/\sigma = \text{SNR} \sim 100$:
\begin{equation}
p_{\text{error}} \approx 2 \Phi(-50) < 10^{-6}
\end{equation}

This is negligible.
