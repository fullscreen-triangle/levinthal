\section{Complexity Analysis}

\subsection{Iteration Count: Exact Formula}

The number of trisection iterations required to reduce search space from $N$ states to 1 is:
\begin{equation}
k = \lceil \log_3 N \rceil
\end{equation}

The ceiling function accounts for the fact that $k$ must be an integer.

\subsubsection{Numerical Examples}

\begin{align}
N = 10^3 &\Rightarrow k = \lceil \log_3 10^3 \rceil = \lceil 6.29 \rceil = 7 \\
N = 10^6 &\Rightarrow k = \lceil \log_3 10^6 \rceil = \lceil 12.6 \rceil = 13 \\
N = 10^{15} &\Rightarrow k = \lceil \log_3 10^{15} \rceil = \lceil 31.5 \rceil = 32
\end{align}

For our experimental system with $N = V_0/\Delta V = (10a_0)^3/(0.01a_0)^3 = 10^9$:
\begin{equation}
k = \lceil \log_3 10^9 \rceil = \lceil 18.9 \rceil = 19
\end{equation}

However, we operate in 1D for simplicity (trisecting only along the radial coordinate), giving $N = 10a_0/0.01a_0 = 10^3$ and $k = 7$.

\subsection{Measurement Count}

Each trisection iteration requires two perturbation measurements (one for $\mathcal{P}_1$, one for $\mathcal{P}_2$). Total measurements:
\begin{equation}
M = 2k = 2 \lceil \log_3 N \rceil
\end{equation}

For $N = 10^3$: $M = 14$ measurements. For $N = 10^{15}$: $M = 64$ measurements.

\subsection{Comparison to Binary Search}

Binary search requires:
\begin{equation}
k_{\text{binary}} = \lceil \log_2 N \rceil, \quad M_{\text{binary}} = \lceil \log_2 N \rceil
\end{equation}

(Only one perturbation per iteration, so measurement count equals iteration count.)

The ratio of measurements is:
\begin{equation}
\frac{M_{\text{ternary}}}{M_{\text{binary}}} = \frac{2 \log_3 N}{\log_2 N} = \frac{2}{\log_2 3} \approx \frac{2}{1.585} \approx 1.26
\end{equation}

Ternary search requires 26\% more measurements than binary search. However, the wall-clock time is not proportional to measurement count but to iteration count (since measurements are parallelized). The iteration ratio is:
\begin{equation}
\frac{k_{\text{ternary}}}{k_{\text{binary}}} = \frac{\log_3 N}{\log_2 N} = \frac{1}{\log_2 3} \approx 0.631
\end{equation}

Ternary search requires 37\% fewer iterations, translating to 37\% faster wall-clock time if iterations have equal duration.

\subsection{Wall-Clock Time}

The wall-clock time is:
\begin{equation}
T_{\text{total}} = k \cdot \tau_{\text{iteration}}
\end{equation}
where $\tau_{\text{iteration}}$ is the time per iteration.

\subsubsection{Iteration Duration}

Each iteration involves:
\begin{enumerate}
\item Perturbation application: $\tau_{\mathcal{P}} = 10^{-8}$ s (both perturbations applied simultaneously, so no doubling)
\item Response measurement: $\tau_{\text{meas}} = 10^{-7}$ s (all five modalities measured in parallel)
\item Data processing: $\tau_{\text{proc}} = 10^{-8}$ s (trit decoding, negligible)
\end{enumerate}

Total: $\tau_{\text{iteration}} \approx \tau_{\mathcal{P}} + \tau_{\text{meas}} \approx 1.1 \times 10^{-7}$ s.

\subsubsection{Total Time}

For ternary search with $k = 7$:
\begin{equation}
T_{\text{ternary}} = 7 \times 1.1 \times 10^{-7} \approx 7.7 \times 10^{-7} \text{ s} = 0.77 \, \mu\text{s}
\end{equation}

For binary search with $k = 11$:
\begin{equation}
T_{\text{binary}} = 11 \times 1.1 \times 10^{-7} \approx 1.2 \times 10^{-6} \text{ s} = 1.2 \, \mu\text{s}
\end{equation}

Speedup: $1.2/0.77 \approx 1.56$ (56\% faster), close to the theoretical $\log_2 3 \approx 1.585$.

\subsection{Best-Case vs Worst-Case}

\subsubsection{Best-Case}

The best case occurs when the target is found in the first trisection step without needing full localization. If only coarse localization (to within one of three regions) is required, $k = 1$ iteration suffices.

More generally, if resolution $\Delta x = L/3^k$ is acceptable, then $k$ iterations suffice. The best case is $k = 1$.

\subsubsection{Worst-Case}

The worst case is when full resolution $\Delta x = \Delta x_{\min}$ is required, needing $k = \log_3(L/\Delta x_{\min})$ iterations.

For our system: $L = 10a_0$, $\Delta x_{\min} = 0.01a_0$, giving $k = \log_3(1000) \approx 6.3$, so $k_{\max} = 7$.

\subsection{Amortized Analysis}

For trajectory tracking, we perform localization at each time step. If the particle moves slowly (distance $\delta x \ll L/3$ per step), we can use the previous localization as a starting point, reducing the search region to a neighborhood around the previous position.

\subsubsection{Incremental Trisection}

If the particle moves by $\delta x = L/27$ (one sub-region), the new search region is $[x_{\text{prev}} - L/9, x_{\text{prev}} + L/9]$ (three adjacent sub-regions). This requires only $k = \log_3 3 = 1$ additional iteration beyond the previous localization depth.

The amortized cost is $O(1)$ per time step (constant, independent of total resolution), assuming slow motion.

\subsection{Comparison to Linear Search}

Linear search examines each region sequentially until the target is found. Complexity: $O(N)$.

For $N = 10^3$:
\begin{itemize}
\item Linear: 500 measurements (average), 1000 (worst)
\item Ternary: 14 measurements
\item Speedup: $500/14 \approx 36\times$ (average), $1000/14 \approx 71\times$ (worst)
\end{itemize}

Ternary search provides exponential speedup over linear search.

\subsection{Quantum Search (Grover) Comparison}

Grover's algorithm achieves $O(\sqrt{N})$ complexity. For $N = 10^3$:
\begin{itemize}
\item Grover: $\sqrt{10^3} \approx 32$ quantum queries
\item Ternary: $14$ classical queries
\end{itemize}

Ternary search outperforms Grover for $N < 10^4$ (due to the $\log N$ vs $\sqrt{N}$ crossover). For larger $N$, Grover is asymptotically faster but requires quantum coherence over $\sqrt{N}$ operations, which is experimentally challenging.

\subsection{Scaling to Large $N$}

\begin{table}[h]
\centering
\begin{tabular}{|c|c|c|c|c|}
\hline
$N$ & Linear & Binary & Ternary & Grover \\
\hline
$10^3$ & 1000 & 11 & 7 & 32 \\
$10^6$ & $10^6$ & 21 & 13 & $10^3$ \\
$10^{15}$ & $10^{15}$ & 50 & 32 & $3 \times 10^7$ \\
\hline
\end{tabular}
\caption{Iteration counts for different search algorithms as a function of search space size $N$.}
\end{table}

Ternary search is optimal for moderate $N$ (classical complexity $\log N$) and experimentally feasible (no quantum coherence required).
