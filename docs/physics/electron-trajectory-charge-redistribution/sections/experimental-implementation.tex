\section{Experimental Implementation}
\label{sec:experimental-implementation}

This section presents the complete experimental protocol for electron trajectory visualization using zero-backaction ternary trisection, following the workflow established in the categorical measurement framework.

\subsection{System Preparation}

\subsubsection{Protein Structure Loading}

The target protein structure is obtained from the Protein Data Bank:
\begin{enumerate}
\item \textbf{Primary target}: Cytochrome c (PDB: 1HRC)
\item \textbf{Secondary targets}: Plastocyanin (PDB: 1PLC), Photosynthetic RC (PDB: 1PRC)
\end{enumerate}

The protein structure provides:
\begin{itemize}
\item Atomic coordinates for all residues
\item Hydrogen bond network topology
\item Metal center geometry (heme iron for cytochrome c)
\item Aromatic residue positions (electron pathway intermediates)
\end{itemize}

\subsubsection{Electron Transfer Pathway Definition}

For cytochrome c, the electron transfer pathway is defined:
\begin{align}
\text{Donor} &: \text{His18 (proximal histidine)} \\
\text{Acceptor} &: \text{Met80 (axial methionine)} \\
\text{Distance} &: R \approx 14~\text{\AA}
\end{align}

The pathway includes intermediate aromatic residues (Phe82, Tyr67, Trp59) that form the superexchange bridge.

\subsection{Categorical Observable Definition}

The categorical observable is constructed from four partition coordinate operators:

\begin{definition}[Partition Coordinate Operators]
The four operators are:
\begin{align}
\hat{n} &: \text{Principal quantum number (radial shell)} \nonumber \\
\hat{\ell} &: \text{Angular momentum quantum number} \nonumber \\
\hat{m} &: \text{Magnetic quantum number} \nonumber \\
\hat{s} &: \text{Spin quantum number} \nonumber
\end{align}
\end{definition}

These operators satisfy the commutation relations:
\begin{equation}
[\hat{n}, \hat{x}] = [\hat{\ell}, \hat{p}] = 0
\end{equation}
ensuring zero-backaction measurement (categorical observables commute with physical observables).

\subsection{Five Spectroscopic Modalities}

\subsubsection{Modality 1: Optical Absorption}

\begin{itemize}
\item \textbf{Wavelength range}: 200--800 nm
\item \textbf{Key bands}: Soret (410 nm), Q-bands (520, 550 nm)
\item \textbf{Measures}: Principal quantum number $n$ via electronic transitions
\item \textbf{Resolution}: 10 fs temporal, 1 nm spectral
\end{itemize}

\subsubsection{Modality 2: Raman Scattering}

\begin{itemize}
\item \textbf{Excitation}: 532 nm (resonance with heme)
\item \textbf{Detection range}: 200--2000 cm$^{-1}$ Stokes shift
\item \textbf{Measures}: Angular momentum $\ell$ via vibrational modes
\item \textbf{Resolution}: 50 fs temporal, 5 cm$^{-1}$ spectral
\end{itemize}

\subsubsection{Modality 3: Magnetic Resonance (EPR)}

\begin{itemize}
\item \textbf{Frequency}: 9.0--9.5 GHz (X-band)
\item \textbf{Field}: 0.32 T
\item \textbf{Measures}: Spin quantum number $s$
\item \textbf{Resolution}: 100 ps temporal
\end{itemize}

\subsubsection{Modality 4: Circular Dichroism}

\begin{itemize}
\item \textbf{Wavelength range}: 190--260 nm (far-UV), 300--700 nm (visible)
\item \textbf{Measures}: Magnetic quantum number $m$ via orbital chirality
\item \textbf{Resolution}: 100 fs temporal
\end{itemize}

\subsubsection{Modality 5: Mass Spectrometry (Time-of-Flight)}

\begin{itemize}
\item \textbf{Mass range}: 1,000--100,000 Da
\item \textbf{Measures}: Temporal evolution $\tau$ via isotope patterns
\item \textbf{Resolution}: 1 ppm mass accuracy
\end{itemize}

\subsection{Perturbation Field Generation}

\subsubsection{Electric Field Gradient (P$_1$)}

The internal electric field gradient arises from charge redistribution:
\begin{equation}
\nabla E_1 = \nabla \left( \int \frac{\rho_e(\mathbf{r}', t)}{|\mathbf{r} - \mathbf{r}'|} d^3r' \right)
\end{equation}

For cytochrome c heme:
\begin{itemize}
\item Field magnitude at heme edge: $E \sim 10^8$ V/m
\item Gradient: $|\nabla E| \sim 10^{15}$ V/m$^2$
\item Direction: Radial from iron center
\end{itemize}

\subsubsection{Magnetic Field Gradient (P$_2$)}

The magnetic field gradient arises from spin-orbit coupling:
\begin{equation}
\nabla B_2 = \nabla \left( \frac{\mu_0}{4\pi} \frac{\boldsymbol{\mu} \times \hat{\mathbf{r}}}{r^2} \right)
\end{equation}

For Fe$^{3+}$ (low-spin, $S = 1/2$):
\begin{itemize}
\item Magnetic moment: $\mu \approx 1~\mu_B$
\item Gradient at 5~\AA: $|\nabla B| \sim 10$ T/m
\end{itemize}

\subsection{Ternary Trisection Protocol}

\subsubsection{Initialization}

\begin{enumerate}
\item Define initial search region: $\Omega^{(0)} = $ protein bounding box
\item Set target resolution: $\Delta r = 0.1$~\AA
\item Initialize backaction counter: $\Delta p_{\text{total}} = 0$
\end{enumerate}

\subsubsection{Iteration Loop}

For each trisection step $k = 1, 2, \ldots$:

\begin{enumerate}
\item \textbf{Apply perturbations}: Activate P$_1$ (electric) and P$_2$ (magnetic) within $\Omega^{(k-1)}$

\item \textbf{Measure responses}:
\begin{align}
r_1 &= \text{response to P}_1 \in \{0, 1\} \\
r_2 &= \text{response to P}_2 \in \{0, 1\}
\end{align}

\item \textbf{Assign trit}:
\begin{equation}
t_k = \begin{cases}
0 & \text{if } (r_1, r_2) = (1, 0) \\
1 & \text{if } (r_1, r_2) = (0, 1) \\
2 & \text{if } (r_1, r_2) = (0, 0)
\end{cases}
\end{equation}

\item \textbf{Update region}: $\Omega^{(k)} = \text{partition}(\Omega^{(k-1)}, t_k)$

\item \textbf{Measure backaction}:
\begin{equation}
\delta_k = \frac{\Delta p_k}{p_0}
\end{equation}
where $\Delta p_k$ is measured via Doppler shift

\item \textbf{Accumulate backaction}: $\Delta p_{\text{total}} += \delta_k$

\item \textbf{Verify zero-backaction}: Assert $\delta_k < 10^{-3}$

\item \textbf{Check convergence}: If $|\Omega^{(k)}| < (\Delta r)^3$, terminate
\end{enumerate}

\subsubsection{Expected Iterations}

For cytochrome c pathway ($R = 14$~\AA, $\Delta r = 0.1$~\AA):
\begin{equation}
k_{\max} = \lceil \log_3(14/0.1)^3 \rceil = \lceil 3 \times \log_3(140) \rceil = 14 \text{ iterations}
\end{equation}

Total measurements: $2 \times 14 = 28$ (two perturbation responses per iteration).

\subsection{Zero-Backaction Verification}

\subsubsection{Backaction Measurement Protocol}

After each trisection step, momentum is measured via three independent methods:

\begin{enumerate}
\item \textbf{Doppler spectroscopy}: Measure Lyman-$\alpha$ line shift
\begin{equation}
\Delta p_{\text{Doppler}} = m_e c \frac{\Delta\lambda}{\lambda_0}
\end{equation}

\item \textbf{Cyclotron frequency}: In Penning trap geometry
\begin{equation}
\Delta p_{\text{cyc}} = eB\Delta\omega_c / \omega_c^2
\end{equation}

\item \textbf{Time-of-flight}: After controlled ejection
\begin{equation}
\Delta p_{\text{TOF}} = m \Delta v = m \frac{\Delta L}{\Delta t}
\end{equation}
\end{enumerate}

\subsubsection{Threshold Criterion}

Zero-backaction is verified when:
\begin{equation}
\frac{\Delta p_{\text{total}}}{p_0} = \sum_{k=1}^{k_{\max}} \delta_k < 10^{-3}
\label{eq:backaction-threshold}
\end{equation}

This is 700$\times$ below the classical backaction limit for comparable spatial resolution.

\subsubsection{Expected Values}

From the zero-backaction measurement framework:
\begin{itemize}
\item Per-step backaction: $\delta_k \sim 10^{-4}$
\item Total backaction (14 steps): $\Delta p_{\text{total}}/p_0 \sim 1.4 \times 10^{-3}$
\item Control (physical measurement): $\Delta p_{\text{phys}}/p_0 \sim 0.8$
\end{itemize}

The ratio demonstrates zero-backaction:
\begin{equation}
\frac{\delta_{\text{categorical}}}{\delta_{\text{physical}}} \sim \frac{10^{-3}}{0.8} \sim 10^{-3}
\end{equation}

\subsection{Categorical Trajectory Extraction}

\subsubsection{Trit Sequence to Partition Coordinates}

The trit sequence $(t_1, t_2, \ldots, t_k)$ is converted to partition coordinates $(n, \ell, m, s)$:

\begin{equation}
n = \sum_{j: j \equiv 0 \pmod{4}} t_j \cdot 3^{\lfloor j/4 \rfloor}
\end{equation}

Similar expressions for $\ell$, $m$, $s$ using trits at positions $j \equiv 1, 2, 3 \pmod{4}$.

\subsubsection{Trajectory Validation}

Each trajectory step must satisfy selection rules:
\begin{align}
\Delta\ell &= \pm 1 \\
|\Delta m| &\leq 1 \\
\Delta s &= 0
\end{align}

Violations indicate measurement error and trigger re-measurement.

\subsection{Wavefunction Reconstruction}

\subsubsection{From Categorical to Physical}

The categorical trajectory defines occupation of nested partitions. Within each partition, the wavefunction is reconstructed:

\begin{equation}
\psi(\mathbf{r}) = \sum_{n,\ell,m,s} c_{n\ell ms} \cdot Y_\ell^m(\theta, \phi) \cdot R_{n\ell}(r) \cdot \chi_s
\end{equation}

where coefficients $c_{n\ell ms}$ are determined by the trit sequence.

\subsubsection{Probability Density}

The electron probability density for visualization:
\begin{equation}
\rho(\mathbf{r}) = |\psi(\mathbf{r})|^2
\end{equation}

\subsection{3D Visualization Rendering}

\subsubsection{Isosurface Representation}

The electron cloud is visualized as isosurfaces:
\begin{itemize}
\item \textbf{Isosurface level}: $\rho = 0.1 \cdot \rho_{\max}$
\item \textbf{Colormap}: Viridis (probability amplitude)
\item \textbf{Opacity}: 50\% (allows seeing internal structure)
\end{itemize}

\subsubsection{Trajectory Overlay}

The categorical trajectory is rendered as:
\begin{itemize}
\item Curve through S-entropy space $(\Sk, \St, \Se)$
\item Color-coded by time (blue $\to$ red)
\item Trit labels at each partition crossing
\end{itemize}

\subsubsection{Backaction Visualization}

Per-step backaction is shown as:
\begin{itemize}
\item Circle diameter proportional to $\delta_k$
\item Color: green ($\delta_k < 10^{-4}$), yellow ($10^{-4} < \delta_k < 10^{-3}$), red ($\delta_k > 10^{-3}$)
\end{itemize}

\subsection{Result Output}

\subsubsection{Saved Data}

The complete experiment produces:
\begin{itemize}
\item Final electron position: $\mathbf{r}^* \pm \Delta r$
\item Complete trajectory: $\{(t_k, \Omega^{(k)}, \delta_k)\}_{k=1}^{k_{\max}}$
\item Categorical states: $\{(n_k, \ell_k, m_k, s_k)\}_{k=1}^{k_{\max}}$
\item Total backaction: $\Delta p_{\text{total}}/p_0$
\item Ternary string: $(t_1, t_2, \ldots, t_{k_{\max}})$
\end{itemize}

\subsubsection{Validation Metrics}

\begin{equation}
\text{Speedup vs binary} = \frac{\log_2 N}{\log_3 N} = \log_2 3 \approx 1.585\times
\end{equation}

For 14 trisection iterations:
\begin{itemize}
\item Ternary: 28 measurements
\item Binary equivalent: $28 \times 1.585 \approx 44$ measurements
\item Reduction: 37\%
\end{itemize}
