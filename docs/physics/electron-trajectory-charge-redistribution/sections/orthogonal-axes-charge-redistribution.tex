\section{Orthogonal Perturbation Axes from Charge Redistribution}
\label{sec:orthogonal-axes}

\subsection{Decomposition into Radial and Angular Components}

In the external perturbation framework, orthogonality is achieved through physically distinct perturbation sources: electric field gradients couple to electric dipole moment while magnetic field gradients couple to magnetic moment. For internal perturbation, we must identify orthogonal components within the single charge redistribution process.

The key observation is that charge redistribution during electron transfer has natural decomposition in spherical coordinates centered on the transfer pathway:

\begin{definition}[Radial Charge Gradient]
The radial component of charge redistribution is:
\begin{equation}
\nabla_r \rhoE = \frac{\partial \rhoE}{\partial r} \hat{\mathbf{r}}
\label{eq:radial-gradient}
\end{equation}
where $r$ is the distance from the transfer axis and $\hat{\mathbf{r}}$ is the radial unit vector.
\end{definition}

\begin{definition}[Angular Charge Gradient]
The angular component of charge redistribution is:
\begin{equation}
\nabla_\theta \rhoE = \frac{1}{r}\frac{\partial \rhoE}{\partial \theta} \hat{\boldsymbol{\theta}}
\label{eq:angular-gradient}
\end{equation}
where $\theta$ is the polar angle and $\hat{\boldsymbol{\theta}}$ is the angular unit vector.
\end{definition}

\subsection{Physical Interpretation}

The two gradient components couple to distinct aspects of electron motion:

\subsubsection{Radial Gradient: Linear Momentum Coupling}

The radial gradient $\nabla_r \rhoE$ measures how electron density changes with distance from the transfer axis. This gradient creates a force:
\begin{equation}
\mathbf{F}_r = -e \nabla_r \Vself = -e \frac{\partial \Vself}{\partial r} \hat{\mathbf{r}}
\label{eq:radial-force}
\end{equation}
that couples to the electron's \emph{linear momentum} along the radial direction.

Physically, the radial gradient is large when:
\begin{itemize}
\item The electron is transitioning between donor and acceptor (charge moving radially)
\item The electron occupies an orbital with radial nodes (s, d orbitals)
\item The protein matrix has radial dielectric inhomogeneity
\end{itemize}

\subsubsection{Angular Gradient: Angular Momentum Coupling}

The angular gradient $\nabla_\theta \rhoE$ measures how electron density changes with angle around the transfer axis. This gradient creates a torque:
\begin{equation}
\boldsymbol{\tau}_\theta = -e \mathbf{r} \times \nabla_\theta \Vself
\label{eq:angular-torque}
\end{equation}
that couples to the electron's \emph{angular momentum} about the transfer axis.

Physically, the angular gradient is large when:
\begin{itemize}
\item The electron is in an orbital with angular nodes (p, d, f orbitals)
\item The transfer pathway has helical character (common in proteins)
\item Nearby aromatic residues create angular anisotropy
\end{itemize}

\subsection{Orthogonality Proof}

\begin{theorem}[Gradient Orthogonality]
\label{thm:orthogonality}
The radial and angular charge redistribution gradients are orthogonal observables:
\begin{equation}
[\nabla_r \rhoE, \nabla_\theta \rhoE] = 0
\label{eq:commutation}
\end{equation}
\end{theorem}

\begin{proof}
The proof proceeds through the tensor product structure of spherical coordinates.

The Hilbert space of electron states decomposes as:
\begin{equation}
\mathcal{H} = \mathcal{H}_r \otimes \mathcal{H}_\theta \otimes \mathcal{H}_\phi
\label{eq:hilbert-decomposition}
\end{equation}
where $\mathcal{H}_r$ is spanned by radial wavefunctions $R_{n\ell}(r)$, $\mathcal{H}_\theta$ by associated Legendre polynomials $P_\ell^m(\cos\theta)$, and $\mathcal{H}_\phi$ by azimuthal phase factors $e^{im\phi}$.

The radial gradient operator acts only on $\mathcal{H}_r$:
\begin{equation}
\nabla_r \rhoE = \frac{\partial}{\partial r} \otimes \mathbf{1}_\theta \otimes \mathbf{1}_\phi
\label{eq:radial-operator}
\end{equation}

The angular gradient operator acts only on $\mathcal{H}_\theta$:
\begin{equation}
\nabla_\theta \rhoE = \mathbf{1}_r \otimes \frac{1}{r}\frac{\partial}{\partial \theta} \otimes \mathbf{1}_\phi
\label{eq:angular-operator}
\end{equation}

Operators acting on different tensor factors commute:
\begin{equation}
[\nabla_r \rhoE, \nabla_\theta \rhoE] = \left[\frac{\partial}{\partial r}, \mathbf{1}_r\right] \otimes \left[\mathbf{1}_\theta, \frac{1}{r}\frac{\partial}{\partial \theta}\right] \otimes [\mathbf{1}_\phi, \mathbf{1}_\phi] = 0
\label{eq:commutator-calculation}
\end{equation}
\end{proof}

\subsection{Three-Outcome Measurement}

The orthogonality ensures that measuring the radial gradient does not disturb the angular gradient, and vice versa. This enables three-outcome measurements:

\begin{enumerate}
\item \textbf{Response to radial gradient only} $(r_1, r_2) = (1, 0)$: Electron is in a region where radial charge redistribution dominates. Assigned trit $t = 0$.

\item \textbf{Response to angular gradient only} $(r_1, r_2) = (0, 1)$: Electron is in a region where angular charge redistribution dominates. Assigned trit $t = 1$.

\item \textbf{No response to either gradient} $(r_1, r_2) = (0, 0)$: Electron is in a region where charge redistribution is minimal (e.g., far from the active transfer region). Assigned trit $t = 2$.
\end{enumerate}

The fourth possibility $(r_1, r_2) = (1, 1)$ is forbidden by orthogonality: if the electron responds strongly to the radial gradient, it must be in a radially-dominated state that has minimal angular gradient response.

\subsection{Spatial Partition Structure}

The three-outcome measurement partitions the transfer region into three zones:

\begin{definition}[Radial Zone]
The radial zone $\Omega_r$ consists of positions where $|\nabla_r \rhoE| > |\nabla_\theta \rhoE|$:
\begin{equation}
\Omega_r = \left\{ \mathbf{r} : \left|\frac{\partial \rhoE}{\partial r}\right| > \frac{1}{r}\left|\frac{\partial \rhoE}{\partial \theta}\right| \right\}
\label{eq:radial-zone}
\end{equation}
This typically includes the donor and acceptor sites and the direct transfer pathway between them.
\end{definition}

\begin{definition}[Angular Zone]
The angular zone $\Omega_\theta$ consists of positions where $|\nabla_\theta \rhoE| > |\nabla_r \rhoE|$:
\begin{equation}
\Omega_\theta = \left\{ \mathbf{r} : \frac{1}{r}\left|\frac{\partial \rhoE}{\partial \theta}\right| > \left|\frac{\partial \rhoE}{\partial r}\right| \right\}
\label{eq:angular-zone}
\end{equation}
This includes regions with curved electron pathways and aromatic intermediates.
\end{definition}

\begin{definition}[Null Zone]
The null zone $\Omega_0$ consists of positions where both gradients are below threshold:
\begin{equation}
\Omega_0 = \left\{ \mathbf{r} : \left|\frac{\partial \rhoE}{\partial r}\right| < \theta_r \text{ and } \frac{1}{r}\left|\frac{\partial \rhoE}{\partial \theta}\right| < \theta_\theta \right\}
\label{eq:null-zone}
\end{equation}
This includes the protein matrix far from the transfer pathway.
\end{definition}

The three zones satisfy $\Omega_r \cup \Omega_\theta \cup \Omega_0 = \mathbb{R}^3$ and are approximately disjoint (overlaps occur only at zone boundaries where gradients are equal).

\subsection{Connection to Partition Coordinates}

The radial and angular gradients connect to partition coordinates $(n, \ell, m, s)$:

\begin{itemize}
\item \textbf{Radial gradient} $\nabla_r \rhoE$ probes the principal quantum number $n$, which determines radial shell structure.

\item \textbf{Angular gradient} $\nabla_\theta \rhoE$ probes the angular momentum quantum numbers $\ell$ and $m$, which determine angular nodal structure.

\item \textbf{Spin} $s$ is probed separately through magnetic detection (EPR), not through charge gradients.
\end{itemize}

This partitioning is consistent with the selection rules derived from the protein folding framework: transitions with $\Delta\ell = \pm 1$ involve changes in angular structure (angular gradient response), while transitions within the same $\ell$ manifold involve radial reorganization (radial gradient response).
