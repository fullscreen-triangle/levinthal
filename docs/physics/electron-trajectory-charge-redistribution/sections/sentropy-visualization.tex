\section{S-Entropy Space Visualization}
\label{sec:visualization}

The trisection algorithm produces a sequence of ternary digits encoding the electron's trajectory. This section develops the mapping from ternary trajectories to three-dimensional S-entropy space, enabling visual representation of electron motion.

\subsection{S-Entropy Coordinate Definitions}

Following the ternary representation framework~\cite{sachikonye2024ternary}, we define three coordinates that characterize the electron's state during transfer:

\begin{definition}[Knowledge Coordinate $\Sk$]
The knowledge coordinate represents the electron's radial position along the transfer pathway:
\begin{equation}
\Sk = \frac{r - r_D}{r_A - r_D} \in [0, 1]
\label{eq:sk-definition}
\end{equation}
where $r$ is the current position, $r_D$ is the donor position, and $r_A$ is the acceptor position.
\end{definition}

At $\Sk = 0$, the electron is at the donor. At $\Sk = 1$, it is at the acceptor. Intermediate values represent positions along the transfer pathway.

\begin{definition}[Temporal Coordinate $\St$]
The temporal coordinate represents the phase in the hydrogen bond oscillation cycle:
\begin{equation}
\St = \frac{\phi_{\text{H-bond}}}{2\pi} \in [0, 1]
\label{eq:st-definition}
\end{equation}
where $\phi_{\text{H-bond}} = \omega_{\text{H}^+} t \mod 2\pi$ is the instantaneous phase.
\end{definition}

The H-bond oscillation frequency $\omega_{\text{H}^+} \sim 4 \times 10^{13}$ rad/s provides a natural ``clock'' for electron transfer, with period $T \sim 150$ fs.

\begin{definition}[Evolution Coordinate $\Se$]
The evolution coordinate represents progress through the reaction coordinate:
\begin{equation}
\Se = \frac{E - E_{\text{TS}}}{E_R - E_{\text{TS}}} \in [0, 1]
\label{eq:se-definition}
\end{equation}
where $E$ is the current energy, $E_{\text{TS}}$ is the transition state energy, and $E_R$ is the reactant energy.
\end{definition}

At $\Se = 0$, the system is at the transition state (maximum energy along the reaction coordinate). At $\Se = 1$, it has relaxed to the reactant or product state.

\subsection{Ternary Address to S-Coordinate Mapping}

Each trisection step refines one S-entropy coordinate by a factor of 3. The mapping from a $k$-trit address $(t_1, t_2, \ldots, t_k)$ to S-coordinates follows a cyclic assignment:

\begin{equation}
\Sk = \sum_{j: j \equiv 0 \pmod{3}} \frac{2t_j + 1}{2 \cdot 3^{\lceil j/3 \rceil}}
\label{eq:sk-from-trits}
\end{equation}

\begin{equation}
\St = \sum_{j: j \equiv 1 \pmod{3}} \frac{2t_j + 1}{2 \cdot 3^{\lceil j/3 \rceil}}
\label{eq:st-from-trits}
\end{equation}

\begin{equation}
\Se = \sum_{j: j \equiv 2 \pmod{3}} \frac{2t_j + 1}{2 \cdot 3^{\lceil j/3 \rceil}}
\label{eq:se-from-trits}
\end{equation}

where the sum runs over trits at positions congruent to 0, 1, or 2 modulo 3.

\subsection{Geometric Interpretation}

The S-entropy space $[0,1]^3$ is partitioned hierarchically by trisection:

\begin{itemize}
\item \textbf{Level 0}: Single cell $[0,1]^3$
\item \textbf{Level 1}: $3^3 = 27$ cells, each of volume $1/27$
\item \textbf{Level 2}: $27^2 = 729$ cells, each of volume $1/729$
\item \textbf{Level $n$}: $27^n$ cells, each of volume $27^{-n}$
\end{itemize}

Each trit refines one coordinate axis by factor 3. After $k$ trits (with $k$ divisible by 3), all three coordinates are refined to precision $3^{-k/3}$.

\subsection{Trajectory Representation}

An electron trajectory through S-entropy space is a curve $\gamma: [0, T] \to [0,1]^3$ parameterized by time:
\begin{equation}
\gamma(t) = (\Sk(t), \St(t), \Se(t))
\label{eq:trajectory-curve}
\end{equation}

The trisection algorithm samples this curve at discrete time points, producing a sequence of S-coordinates:
\begin{equation}
\{\gamma(t_1), \gamma(t_2), \ldots, \gamma(t_N)\}
\label{eq:sampled-trajectory}
\end{equation}

Interpolating between samples yields a continuous trajectory visualization.

\subsection{Visualization Rendering}

\subsubsection{3D Trajectory Plot}

The primary visualization is a 3D curve in $[0,1]^3$ with:
\begin{itemize}
\item \textbf{Position}: $(\Sk, \St, \Se)$ coordinates
\item \textbf{Color}: Time encoded as color gradient (blue $\to$ red)
\item \textbf{Width}: Confidence (wider = higher confidence)
\item \textbf{Markers}: Key states (donor, transition state, acceptor)
\end{itemize}

\subsubsection{Partition Grid Overlay}

The hierarchical partition structure is rendered as a wireframe:
\begin{itemize}
\item \textbf{Level 1 boundaries}: Thick lines dividing $[0,1]^3$ into 27 cells
\item \textbf{Level 2 boundaries}: Medium lines for 729 cells
\item \textbf{Level 3+ boundaries}: Thin lines for finer partitions
\end{itemize}

The trajectory crosses partition boundaries at each trisection step.

\subsubsection{Phase-Lock Overlay}

The Kuramoto order parameter $r(t)$ is encoded as:
\begin{itemize}
\item \textbf{Background color}: High $r$ (synchronized) = blue, low $r$ (desynchronized) = red
\item \textbf{Isosurface}: Surface of constant $r = 0.8$ (critical coherence threshold)
\item \textbf{Vector field}: Local phase gradient showing H-bond coupling direction
\end{itemize}

\subsection{Example: Cytochrome c Electron Transfer}

For cytochrome c electron transfer ($R = 14$~\AA, $k_{\text{ET}} = 10^5$ s$^{-1}$):

\begin{enumerate}
\item \textbf{Initial state} ($t = 0$): $(\Sk, \St, \Se) = (0, 0.25, 1)$
\begin{itemize}
\item Electron at donor ($\Sk = 0$)
\item H-bond phase at quarter-cycle ($\St = 0.25$)
\item System in reactant well ($\Se = 1$)
\end{itemize}

\item \textbf{Transition state} ($t = \tau_{\text{TS}}$): $(\Sk, \St, \Se) = (0.5, 0.75, 0)$
\begin{itemize}
\item Electron at pathway midpoint ($\Sk = 0.5$)
\item H-bond at three-quarter cycle ($\St = 0.75$)
\item System at transition state ($\Se = 0$)
\end{itemize}

\item \textbf{Final state} ($t = \tau_{\text{ET}}$): $(\Sk, \St, \Se) = (1, 0.5, 1)$
\begin{itemize}
\item Electron at acceptor ($\Sk = 1$)
\item H-bond at half-cycle ($\St = 0.5$, $\pi$ phase shift)
\item System in product well ($\Se = 1$)
\end{itemize}
\end{enumerate}

The trajectory forms a curved path through S-entropy space, with the characteristic ``dip'' to $\Se = 0$ at the transition state.

\subsection{Resolution and Precision}

The precision of S-entropy coordinates is determined by the number of trisection steps:

\begin{equation}
\Delta S = 3^{-k/3}
\label{eq:sentropy-precision}
\end{equation}

For $k = 12$ trisection steps (24 measurements):
\begin{equation}
\Delta S = 3^{-4} = 1/81 \approx 0.012
\label{eq:precision-example}
\end{equation}

This corresponds to spatial resolution of $\sim 0.17$~\AA\ for a 14~\AA\ transfer pathway.

\subsection{Animation and Dynamics}

Time-resolved visualization shows the electron's motion as an animated trajectory:

\begin{enumerate}
\item \textbf{Frame rate}: 30 fps for playback
\item \textbf{Time compression}: 1 \textmu s of transfer $\to$ 1 second of animation (10$^6\times$ speedup)
\item \textbf{Trajectory growth}: Curve extends as new trisection data arrives
\item \textbf{Confidence evolution}: Uncertainty shrinks as localization proceeds
\end{enumerate}

For the photosynthetic reaction center (3 ps transfer), a 1-second animation shows the complete primary electron transfer with $\sim 10^{11}\times$ speedup.
