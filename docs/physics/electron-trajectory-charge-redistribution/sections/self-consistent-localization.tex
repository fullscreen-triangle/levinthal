\section{Self-Consistent Localization Algorithm}
\label{sec:self-consistent}

\subsection{The Bootstrap Problem}

The internal perturbation framework faces a conceptual challenge: the electron's position determines the perturbation field, but we need the perturbation field to determine the electron's position. This circular dependence is the ``bootstrap problem'' of self-localization.

Formally, let $\mathbf{r}(t)$ be the electron position at time $t$. The self-potential depends on this position:
\begin{equation}
\Vself(\mathbf{r}, t) = \Vself[\rhoE(\mathbf{r}', t); \mathbf{r}] = \int \frac{\rhoE(\mathbf{r}', t)}{|\mathbf{r} - \mathbf{r}'|} d^3r'
\label{eq:self-potential-circular}
\end{equation}

But $\rhoE(\mathbf{r}', t) = |\psi(\mathbf{r}', t)|^2$ depends on the wavefunction, which evolves under the Hamiltonian containing $\Vself$. The circularity is:
\begin{equation}
\mathbf{r}(t) \to \rhoE \to \Vself \to \hat{H} \to \psi \to \rhoE' \to \cdots
\label{eq:circular-chain}
\end{equation}

\subsection{Resolution via Temporal Retardation}

The bootstrap problem is resolved by recognizing that the perturbation at position $\mathbf{r}$ at time $t$ is generated by the electron density at \emph{earlier} time $t - \tauret$:
\begin{equation}
\Vself(\mathbf{r}, t) = \int \frac{\rhoE(\mathbf{r}', t - \tauret)}{|\mathbf{r} - \mathbf{r}'|} d^3r'
\label{eq:retarded-potential}
\end{equation}

The retardation time $\tauret = |\mathbf{r} - \mathbf{r}'|/c \sim 5$ attoseconds breaks the circular dependence: the perturbation at time $t$ depends on the \emph{past} electron position, not the current position being localized.

This is analogous to how classical electromagnetism handles self-force: the Li\'enard-Wiechert potentials express fields in terms of retarded source positions, avoiding instantaneous self-interaction paradoxes.

\subsection{Iterative Self-Consistent Field Algorithm}

Given the retarded structure, localization proceeds through an iterative algorithm:

\begin{algorithm}
\caption{Self-Consistent Trisection Localization}
\label{alg:self-consistent}
\begin{algorithmic}[1]
\REQUIRE Transfer pathway $\Omega$, initial electron density $\rhoE^{(0)}$
\ENSURE Localized position $\mathbf{r}^*$ with resolution $\Delta r$
\STATE Initialize: $\Omega^{(0)} \leftarrow \Omega$ (full pathway)
\STATE Initialize: $\rhoE^{(0)}(\mathbf{r}) \leftarrow \text{uniform over } \Omega^{(0)}$
\FOR{$k = 1, 2, \ldots, k_{\max}$}
    \STATE Compute retarded self-potential: $\Vself^{(k)} \leftarrow \Vself[\rhoE^{(k-1)}]$
    \STATE Compute gradients: $\nabla_r \Vself^{(k)}$, $\nabla_\theta \Vself^{(k)}$
    \STATE Determine response: $(r_1, r_2) \leftarrow$ spectroscopic measurement
    \STATE Assign trit: $t_k \leftarrow$ encode$(r_1, r_2)$ \COMMENT{Eq.~\eqref{eq:trit-assignment}}
    \STATE Update region: $\Omega^{(k)} \leftarrow$ partition$(\Omega^{(k-1)}, t_k)$
    \STATE Update density: $\rhoE^{(k)}(\mathbf{r}) \leftarrow \rhoE^{(k-1)}(\mathbf{r}) \cdot \mathbf{1}_{\Omega^{(k)}}(\mathbf{r})$
    \STATE Normalize: $\rhoE^{(k)} \leftarrow \rhoE^{(k)} / \int_{\Omega^{(k)}} \rhoE^{(k)} d^3r$
    \IF{$|\Omega^{(k)}| < (\Delta r)^3$}
        \STATE \textbf{break} \COMMENT{Desired resolution achieved}
    \ENDIF
\ENDFOR
\STATE $\mathbf{r}^* \leftarrow$ centroid$(\Omega^{(k)})$
\RETURN $\mathbf{r}^*$, trajectory $(t_1, t_2, \ldots, t_k)$
\end{algorithmic}
\end{algorithm}

\subsection{Convergence Analysis}

\begin{theorem}[Convergence of Self-Consistent Trisection]
\label{thm:convergence}
The self-consistent trisection algorithm converges in $k = \lceil \log_3(V_0/(\Delta r)^3) \rceil$ iterations, where $V_0$ is the initial search volume and $\Delta r$ is the target resolution.
\end{theorem}

\begin{proof}
At each iteration, the search volume is reduced by a factor of 3:
\begin{equation}
|\Omega^{(k)}| = \frac{|\Omega^{(k-1)}|}{3} = \frac{V_0}{3^k}
\label{eq:volume-reduction}
\end{equation}

The algorithm terminates when $|\Omega^{(k)}| < (\Delta r)^3$:
\begin{equation}
\frac{V_0}{3^k} < (\Delta r)^3 \implies k > \log_3\left(\frac{V_0}{(\Delta r)^3}\right)
\label{eq:termination-condition}
\end{equation}

The minimum integer satisfying this is $k = \lceil \log_3(V_0/(\Delta r)^3) \rceil$.
\end{proof}

For a transfer pathway of length $L = 15$~\AA\ and cross-section $A = (5~\text{\AA})^2 = 25$~\AA$^2$:
\begin{equation}
V_0 = L \times A = 375~\text{\AA}^3
\label{eq:initial-volume}
\end{equation}

For resolution $\Delta r = 0.1$~\AA\ (comparable to bond length precision):
\begin{equation}
k = \lceil \log_3(375 / 0.001) \rceil = \lceil \log_3(375000) \rceil = \lceil 11.7 \rceil = 12 \text{ iterations}
\label{eq:iteration-count}
\end{equation}

Each iteration involves 2 measurements (radial and angular gradients), for a total of 24 spectroscopic measurements.

\subsection{Stability Analysis}

The self-consistent iteration is stable because the perturbation-to-density mapping is a contraction:

\begin{lemma}[Contraction Property]
\label{lem:contraction}
Let $\|\cdot\|$ denote the $L^2$ norm on density space. The mapping $\mathcal{T}: \rho_e^{(k-1)} \mapsto \rho_e^{(k)}$ satisfies:
\begin{equation}
\|\mathcal{T}(\rho_{e,1}) - \mathcal{T}(\rho_{e,2})\| \leq \alpha \|\rho_{e,1} - \rho_{e,2}\|
\label{eq:contraction}
\end{equation}
with contraction factor $\alpha = 1/3 < 1$.
\end{lemma}

\begin{proof}
The trisection operation partitions the support of $\rhoE$ into three regions and retains one. The retained region has $1/3$ the volume of the original, so:
\begin{equation}
\|\mathcal{T}(\rhoE)\|^2 = \int_{\Omega^{(k)}} |\rhoE|^2 d^3r \leq \frac{1}{3} \int_{\Omega^{(k-1)}} |\rhoE|^2 d^3r = \frac{1}{3} \|\rhoE\|^2
\label{eq:norm-reduction}
\end{equation}

The contraction follows from the linearity of trisection within each partition.
\end{proof}

By the Banach fixed-point theorem, the iteration converges to a unique fixed point---the localized electron density.

\subsection{Error Analysis}

Two sources of error affect localization accuracy:

\subsubsection{Measurement Noise}

If the spectroscopic measurements have noise level $\sigma_m$, the trit assignment may be incorrect with probability:
\begin{equation}
p_{\text{error}} \approx \Phi\left(-\frac{\theta - \mu}{\sigma_m}\right)
\label{eq:error-probability}
\end{equation}
where $\theta$ is the detection threshold, $\mu$ is the signal level, and $\Phi$ is the Gaussian CDF.

For high signal-to-noise ratio ($\mu/\sigma_m > 10$), $p_{\text{error}} < 10^{-5}$ per measurement. With 24 measurements per localization, the total error probability is $\lesssim 2 \times 10^{-4}$.

\subsubsection{Retardation Approximation}

The retardation time $\tauret \sim 5$ attoseconds is neglected compared to the measurement timescale ($\sim 10$ femtoseconds). The error from this approximation is:
\begin{equation}
\delta r \sim v_e \cdot \tauret \sim \frac{\hbar k}{m_e} \cdot \tauret
\label{eq:retardation-error}
\end{equation}

For typical electron momenta $\hbar k \sim 1$~eV/c:
\begin{equation}
\delta r \sim \frac{1 \text{ eV}}{0.5 \text{ MeV}} \times c \times 5 \times 10^{-18} \text{ s} \sim 3 \times 10^{-15} \text{ m} = 3 \text{ fm}
\label{eq:retardation-error-value}
\end{equation}

This is negligible compared to the target resolution of 0.1~\AA\ = 10 fm.
