\section{Testable Predictions}
\label{sec:predictions}

The internal perturbation framework makes five quantitative predictions that distinguish it from Marcus theory and conventional electron transfer descriptions. Each prediction specifies an experimental test with expected outcomes.

\subsection{Prediction 1: Trajectory Determinism}

\subsubsection{Statement}

Electron transfer trajectories are deterministic, not stochastic. The trajectory variance across identically prepared systems is:
\begin{equation}
\sigma^2_{\text{traj}} = \frac{1}{N}\sum_{i=1}^{N} \|\gamma_i - \bar{\gamma}\|^2 < 10^{-6}
\label{eq:trajectory-variance}
\end{equation}
where $\gamma_i$ is the $i$-th measured trajectory, $\bar{\gamma}$ is the mean trajectory, and $\|\cdot\|$ is the $L^2$ norm.

\subsubsection{Contrast with Marcus Theory}

Marcus theory predicts stochastic electron transfer with exponential waiting time distribution:
\begin{equation}
P(t) = k_{\text{ET}} e^{-k_{\text{ET}} t}
\label{eq:marcus-waiting-time}
\end{equation}

This implies large trajectory variance: different transfer events follow different paths through the activation barrier region.

\subsubsection{Experimental Test}

\begin{enumerate}
\item Prepare $N > 100$ identical protein samples with electron at donor
\item Trigger electron transfer simultaneously (flash photolysis or rapid mixing)
\item Measure trajectory using multi-modal spectroscopy
\item Compute trajectory variance across samples
\end{enumerate}

\textbf{Expected result}: Variance $\sigma^2 < 10^{-6}$ (our framework) vs.\ $\sigma^2 \sim 0.1$--1 (Marcus theory).

\subsection{Prediction 2: H-Bond Frequency Modulation}

\subsubsection{Statement}

Hydrogen bond vibrational frequencies shift by 1--5\% during electron transfer, with spatial correlation to electron position:
\begin{equation}
\omega_{\text{H-bond}}(\mathbf{r}, t) = \omega_0 \left(1 + \alpha \frac{\rhoE(\mathbf{r}, t)}{\rho_0}\right)
\label{eq:hbond-modulation}
\end{equation}
where $\alpha \approx 0.05$ is the coupling coefficient.

The correlation between H-bond frequency shift and electron position satisfies:
\begin{equation}
R^2(\Delta\omega, \Sk) > 0.8
\label{eq:hbond-correlation}
\end{equation}

\subsubsection{Contrast with Standard Theory}

The Born-Oppenheimer approximation assumes electrons move much faster than nuclei, implying nuclear (H-bond) frequencies are independent of instantaneous electron position. Standard theory predicts $R^2 \approx 0$.

\subsubsection{Experimental Test}

\begin{enumerate}
\item Monitor H-bond stretch frequency (3300 cm$^{-1}$) using transient IR
\item Simultaneously measure electron position using optical absorption
\item Compute correlation between $\Delta\omega_{\text{H-bond}}$ and $\Sk$
\end{enumerate}

\textbf{Expected result}: $R^2 > 0.8$ (our framework) vs.\ $R^2 \approx 0$ (Born-Oppenheimer).

\subsection{Prediction 3: Coherence Dip During Transfer}

\subsubsection{Statement}

The Kuramoto order parameter $r(t)$ for the hydrogen bond network shows a characteristic dip during electron transfer:
\begin{equation}
r(t) = \left| \frac{1}{N_{\text{H-bond}}} \sum_{j=1}^{N_{\text{H-bond}}} e^{i\phi_j(t)} \right|
\label{eq:kuramoto-order-parameter}
\end{equation}

The dip magnitude is:
\begin{equation}
\Delta r = r_{\text{baseline}} - r_{\text{min}} \in [0.1, 0.3]
\label{eq:coherence-dip}
\end{equation}

The recovery time is:
\begin{equation}
\tau_{\text{recovery}} \in [1, 10] \text{ ps}
\label{eq:recovery-time}
\end{equation}

\subsubsection{Contrast with Standard Theory}

Standard electron transfer theory makes no prediction about H-bond network coherence. The Kuramoto order parameter is not a recognized observable in Marcus theory.

\subsubsection{Experimental Test}

\begin{enumerate}
\item Compute effective order parameter from multi-site H-bond frequency coherence (measured via 2D-IR)
\item Track $r(t)$ before, during, and after electron transfer
\item Measure dip magnitude $\Delta r$ and recovery time $\tau_{\text{recovery}}$
\end{enumerate}

\textbf{Expected result}: $\Delta r = 0.1$--0.3 with $\tau_{\text{recovery}} = 1$--10 ps (our framework) vs.\ no prediction (Marcus theory).

\subsection{Prediction 4: Selection Rule Enforcement}

\subsubsection{Statement}

Electron transfer obeys selection rules derived from partition coordinate constraints:
\begin{align}
\Delta\ell &= \pm 1 \\
|\Delta m| &\leq 1 \\
\Delta s &= 0
\label{eq:selection-rules}
\end{align}

The enforcement ratio is:
\begin{equation}
\frac{\Gamma_{\text{allowed}}}{\Gamma_{\text{forbidden}}} > 10^8
\label{eq:enforcement-ratio}
\end{equation}
where $\Gamma$ is the transition rate.

\subsubsection{Contrast with Standard Theory}

Marcus theory imposes no selection rules on angular momentum. The Franck-Condon principle governs vibrational overlap but not orbital angular momentum changes.

\subsubsection{Experimental Test}

\begin{enumerate}
\item Design ligand modifications that would enable $\Delta\ell = 0$ or $\Delta\ell = \pm 2$ transitions
\item Measure electron transfer rates for allowed ($\Delta\ell = \pm 1$) and forbidden ($\Delta\ell \neq \pm 1$) pathways
\item Compute rate ratio
\end{enumerate}

\textbf{Expected result}: $\Gamma_{\text{allowed}}/\Gamma_{\text{forbidden}} > 10^8$ (our framework) vs.\ comparable rates (Marcus theory, which allows any $\Delta\ell$).

\subsection{Prediction 5: $O(\log_3 N)$ Localization Complexity}

\subsubsection{Statement}

The number of measurements required to localize the electron to precision $\Delta r$ scales as:
\begin{equation}
N_{\text{meas}} = c \cdot \log_3\left(\frac{R}{\Delta r}\right)
\label{eq:localization-complexity}
\end{equation}
where $R$ is the transfer distance, $\Delta r$ is the target resolution, and $c \approx 6$ (accounting for two measurements per trisection step across three coordinates).

\subsubsection{Contrast with Standard Theory}

Conventional position measurement would require:
\begin{equation}
N_{\text{conv}} \propto \left(\frac{R}{\Delta r}\right)^3
\label{eq:conventional-complexity}
\end{equation}
measurements to scan the full 3D transfer region.

\subsubsection{Experimental Test}

\begin{enumerate}
\item Localize electron position at varying precisions $\Delta r$
\item Count total number of spectroscopic measurements required
\item Fit to $N_{\text{meas}} = a \log_3(R/\Delta r) + b$
\end{enumerate}

\textbf{Expected result}: Logarithmic scaling (our framework) vs.\ cubic scaling (conventional).

For $R = 15$~\AA\ and $\Delta r = 0.1$~\AA:
\begin{itemize}
\item Our prediction: $N_{\text{meas}} \approx 6 \times \log_3(150) \approx 6 \times 4.6 \approx 28$ measurements
\item Conventional: $N_{\text{conv}} \propto 150^3 \sim 3 \times 10^6$ measurements
\end{itemize}

The factor of $\sim 10^5$ difference is experimentally distinguishable.

\subsection{Summary of Predictions}

\begin{table}[h]
\centering
\caption{Summary of testable predictions comparing internal perturbation framework to Marcus theory.}
\label{tab:predictions-summary}
\begin{tabular}{@{}lll@{}}
\toprule
Prediction & Our Framework & Marcus Theory \\
\midrule
Trajectory variance & $\sigma^2 < 10^{-6}$ & $\sigma^2 \sim 0.1$--1 \\
H-bond/position correlation & $R^2 > 0.8$ & $R^2 \approx 0$ \\
Coherence dip & $\Delta r = 0.1$--0.3 & No prediction \\
Selection rule ratio & $> 10^8$ & $\sim 1$ \\
Localization scaling & $O(\log N)$ & $O(N^3)$ \\
\bottomrule
\end{tabular}
\end{table}

All five predictions are experimentally accessible with current ultrafast spectroscopy techniques on the target protein systems described in Section~\ref{sec:target-systems}.
