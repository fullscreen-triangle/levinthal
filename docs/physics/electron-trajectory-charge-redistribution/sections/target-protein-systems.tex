\section{Target Protein Systems}
\label{sec:target-systems}

Three protein systems are selected for experimental validation, chosen for their well-characterized electron transfer pathways, strong spectroscopic signatures, and increasing complexity.

\subsection{Blue Copper Proteins: Plastocyanin and Azurin}

\subsubsection{System Description}

Blue copper proteins contain a single type 1 copper site that cycles between Cu$^{2+}$ (oxidized, paramagnetic) and Cu$^+$ (reduced, diamagnetic) states. The copper is coordinated by two histidine nitrogens, one cysteine sulfur, and one weakly bound methionine sulfur in a distorted tetrahedral geometry.

Key parameters:
\begin{itemize}
\item \textbf{Redox potential}: $E^0 \approx +350$ mV vs.\ SHE
\item \textbf{Reorganization energy}: $\lambda \approx 0.7$ eV
\item \textbf{Electronic coupling}: $H_{DA} \approx 10^{-3}$ eV
\item \textbf{Transfer rate}: $k_{\text{ET}} \sim 10^3$--$10^5$ s$^{-1}$
\end{itemize}

\subsubsection{Advantages for Internal Perturbation}

Plastocyanin offers the simplest case for internal perturbation visualization:

\begin{enumerate}
\item \textbf{Single metal site}: The electron transfers to/from a single copper center, avoiding complications from multi-site systems.

\item \textbf{Strong EPR signal}: Cu$^{2+}$ has an unpaired electron with characteristic $g$-tensor anisotropy ($g_\parallel \approx 2.2$, $g_\perp \approx 2.05$), enabling spin-state detection.

\item \textbf{Intense optical absorption}: The Cu--S$_{\text{Cys}}$ charge transfer band at 600~nm ($\epsilon \sim 5000$ M$^{-1}$cm$^{-1}$) provides strong optical signal.

\item \textbf{Small protein size}: Plastocyanin (99 residues, 10.5 kDa) and azurin (128 residues, 14 kDa) are small enough for high-resolution structural studies.
\end{enumerate}

\subsubsection{Charge Redistribution Characteristics}

During Cu$^{2+} \to$ Cu$^+$ reduction:
\begin{itemize}
\item Electron density shifts from protein surface to copper $d_{x^2-y^2}$ orbital
\item Electric field change at copper: $\Delta E \sim 10^8$ V/m
\item Radial gradient dominated by Cu--ligand bond redistribution
\item Angular gradient from $d$-orbital shape change
\end{itemize}

\subsection{Cytochrome c: Heme Iron Electron Transfer}

\subsubsection{System Description}

Cytochrome c is a 104-residue heme protein that shuttles electrons between Complex III and Complex IV in the mitochondrial respiratory chain. The heme iron cycles between Fe$^{3+}$ (ferric, low-spin $S = 1/2$) and Fe$^{2+}$ (ferrous, diamagnetic $S = 0$) states.

Key parameters:
\begin{itemize}
\item \textbf{Redox potential}: $E^0 \approx +260$ mV vs.\ SHE
\item \textbf{Transfer distance}: $R \approx 14$~\AA\ (edge-to-edge between heme groups)
\item \textbf{Transfer pathway}: Aromatic residues (Trp, Phe, Tyr) form superexchange bridges
\item \textbf{Transfer rate}: $k_{\text{ET}} \sim 10^5$--$10^6$ s$^{-1}$
\end{itemize}

\subsubsection{Advantages for Internal Perturbation}

Cytochrome c offers intermediate complexity with rich spectroscopic signatures:

\begin{enumerate}
\item \textbf{Multiple spectroscopic probes}: Soret band (410 nm), Q-bands (520, 550 nm), and charge transfer bands provide optical handles.

\item \textbf{EPR-active oxidized state}: Fe$^{3+}$ low-spin has characteristic EPR ($g_z \approx 3.0$, $g_y \approx 2.2$, $g_x \approx 1.2$).

\item \textbf{Defined transfer pathway}: Crystal structures of cytochrome c--cytochrome c peroxidase complexes reveal the electron pathway through aromatic residues.

\item \textbf{Hydrogen bond involvement}: The heme propionates form H-bonds with nearby residues, enabling coupling to H-bond oscillations.
\end{enumerate}

\subsubsection{Charge Redistribution Characteristics}

During Fe$^{3+} \to$ Fe$^{2+}$ reduction:
\begin{itemize}
\item Electron density redistributes across the porphyrin $\pi$-system
\item Electric field change at heme periphery: $\Delta E \sim 10^7$ V/m
\item Strong angular gradients from porphyrin nodal structure
\item Pathway intermediates (Trp, Tyr) show transient charge accumulation
\end{itemize}

\subsection{Photosynthetic Reaction Center: Ultrafast Cascade}

\subsubsection{System Description}

The bacterial photosynthetic reaction center (RC) performs light-driven electron transfer through a multi-step cascade:

\begin{equation}
\text{P}^* \xrightarrow{3 \text{ ps}} \text{P}^+\text{BPhe}^- \xrightarrow{200 \text{ ps}} \text{P}^+\text{Q}_A^- \xrightarrow{200 \text{ \textmu s}} \text{P}^+\text{Q}_B^-
\label{eq:rc-cascade}
\end{equation}

where P is the special pair (bacteriochlorophyll dimer), BPhe is bacteriopheophytin, and Q$_A$/Q$_B$ are quinones.

Key parameters:
\begin{itemize}
\item \textbf{Total transfer distance}: $\sim 23$~\AA\ from P to Q$_A$
\item \textbf{Initial step (P$^* \to$ BPhe)}: 3 ps, fastest known biological ET
\item \textbf{Quantum efficiency}: $>99\%$
\item \textbf{Reorganization energy}: $\lambda \approx 0.25$ eV (unusually small)
\end{itemize}

\subsubsection{Advantages for Internal Perturbation}

The photosynthetic RC provides the most demanding test case:

\begin{enumerate}
\item \textbf{Ultrafast timescales}: The 3 ps primary step requires femtosecond time resolution.

\item \textbf{Multiple intermediates}: The cascade through BPhe, Q$_A$, Q$_B$ provides multiple localization checkpoints.

\item \textbf{Near-unity efficiency}: The high quantum yield implies deterministic trajectory (consistent with our framework).

\item \textbf{Defined structure}: High-resolution crystal structures (1.8~\AA) reveal atomic details of the transfer pathway.
\end{enumerate}

\subsubsection{Charge Redistribution Characteristics}

During the P$^* \to$ BPhe$^-$ step:
\begin{itemize}
\item Electron density shifts across 10~\AA\ in 3 ps
\item Transient electric field: $\Delta E \sim 10^9$ V/m
\item Strong radial gradient along the L-branch pathway
\item Angular gradients from chlorophyll ring geometry
\end{itemize}

\subsection{Comparative Analysis}

\begin{table}[h]
\centering
\caption{Comparison of target protein systems for internal perturbation electron trajectory visualization.}
\label{tab:system-comparison}
\begin{tabular}{@{}lccc@{}}
\toprule
Property & Blue Copper & Cytochrome c & Reaction Center \\
\midrule
Transfer distance (\AA) & $<5$ & 14 & 23 \\
Transfer time & \textmu s--ms & \textmu s & 3 ps \\
Number of sites & 1 & 1 (via pathway) & 4 \\
$\Delta E$ (V/m) & $10^8$ & $10^7$ & $10^9$ \\
EPR signal & Strong & Medium & Weak \\
Optical signal & Strong & Strong & Very strong \\
Complexity & Low & Medium & High \\
\bottomrule
\end{tabular}
\end{table}

\subsection{Recommended Experimental Sequence}

We recommend starting with blue copper proteins (simplest case), proceeding to cytochrome c (intermediate pathway complexity), and culminating with the photosynthetic reaction center (ultrafast, multi-step cascade). This progression allows method development on tractable systems before tackling the most challenging target.
