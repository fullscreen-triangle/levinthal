\section{Ternary State Trajectory Completion through Poincar\'e Recurrence}

\subsection{Trajectory in S-Entropy Space}

\begin{definition}[S-Entropy Coordinates]
For molecular system, define three-dimensional entropy coordinate space $\Sspace = [0,1]^3$ with coordinates:
\begin{itemize}
\item $\Sk \in [0,1]$: Knowledge entropy (categorical identity - which mode/state)
\item $\St \in [0,1]$: Temporal entropy (phase - position in cycle)
\item $\Se \in [0,1]$: Evolutionary entropy (quantum number - energy level)
\end{itemize}
\end{definition}

\begin{theorem}[Trajectory Boundedness]
\label{thm:trajectory_bounded}
Molecular state trajectory in $\Sspace$ is bounded: $(\Sk(t), \St(t), \Se(t)) \in [0,1]^3$ for all $t$. This follows from normalization of entropy coordinates.
\end{theorem}

\begin{proof}
Each entropy coordinate is defined as normalized quantity:
\begin{align}
\Sk &= \frac{\text{mode index}}{N_{\text{modes}}} \in [0,1] \\
\St &= \frac{\phi}{2\pi} \in [0,1] \quad \text{(phase normalized to cycle)} \\
\Se &= \frac{v}{v_{\max}} \in [0,1] \quad \text{(quantum number normalized to maximum)}
\end{align}

Since each coordinate is bounded to $[0,1]$, the trajectory $\mathbf{s}(t) = (\Sk(t), \St(t), \Se(t))$ remains in unit cube $\Sspace = [0,1]^3$ for all time.

Boundedness ensures Poincar\'e recurrence applies: trajectory returns arbitrarily close to initial conditions infinitely often.
\end{proof}

\subsection{Coupled Rate Equations for Ternary States}

\begin{theorem}[Ternary State Evolution]
\label{thm:ternary_evolution}
For molecular system with electronic states $\{|0\rangle, |1\rangle, |2\rangle\}$ and amplitudes $\{c_0(t), c_1(t), c_2(t)\}$, the evolution satisfies coupled rate equations:
\begin{align}
\frac{dc_2}{dt} &= -\frac{c_2}{\tau_{\text{em}}} - \frac{c_2}{\tau_{\text{vib}}} \\
\frac{dc_1}{dt} &= \frac{c_2}{\tau_{\text{vib}}} - \frac{c_1}{\tau_{\text{vib}}} \\
\frac{dc_0}{dt} &= \frac{c_2}{\tau_{\text{em}}} + \frac{c_1}{\tau_{\text{vib}}}
\end{align}
with normalization $c_0^2 + c_1^2 + c_2^2 = 1$.
\end{theorem}

\begin{proof}
State $|2\rangle$ (excited) decays through two channels: radiative emission ($\tau_{\text{em}}$) to $|0\rangle$ and vibrational relaxation ($\tau_{\text{vib}}$) to $|1\rangle$. Rate equation:
\begin{equation}
\frac{dc_2}{dt} = -\left(\frac{1}{\tau_{\text{em}}} + \frac{1}{\tau_{\text{vib}}}\right) c_2
\end{equation}

State $|1\rangle$ (natural) gains population from $|2\rangle$ relaxation and loses to $|0\rangle$ through vibrational cooling:
\begin{equation}
\frac{dc_1}{dt} = \frac{c_2}{\tau_{\text{vib}}} - \frac{c_1}{\tau_{\text{vib}}}
\end{equation}

State $|0\rangle$ (ground) gains from both $|2\rangle$ emission and $|1\rangle$ relaxation:
\begin{equation}
\frac{dc_0}{dt} = \frac{c_2}{\tau_{\text{em}}} + \frac{c_1}{\tau_{\text{vib}}}
\end{equation}

Sum of rates: $\frac{d}{dt}(c_0 + c_1 + c_2) = 0$, confirming normalization conservation.

Initial condition: $c_2(0) = 1, c_1(0) = 0, c_0(0) = 0$ (pure excited state after UV excitation).
Final state: $c_2(\infty) = 0, c_1(\infty) = 0, c_0(\infty) = 1$ (complete relaxation to ground).
\end{proof}

\subsection{Trajectory Fidelity}

\begin{definition}[State Fidelity]
For measured state $|\psi_{\text{meas}}\rangle = \sum_i c_i^{\text{meas}} |i\rangle$ and theoretical state $|\psi_{\text{theory}}\rangle = \sum_i c_i^{\text{theory}} |i\rangle$, the fidelity is:
\begin{equation}
F = |\langle \psi_{\text{meas}} | \psi_{\text{theory}} \rangle|^2 = \left|\sum_i (c_i^{\text{meas}})^* c_i^{\text{theory}}\right|^2
\end{equation}
For real amplitudes, $F = \left(\sum_i c_i^{\text{meas}} c_i^{\text{theory}}\right)^2$.
\end{definition}

\begin{theorem}[Average Fidelity Bound]
\label{thm:fidelity_bound}
For trajectory over time interval $[0, T]$, the average fidelity is:
\begin{equation}
\bar{F} = \frac{1}{T}\int_0^T F(t) dt
\end{equation}
High-fidelity reconstruction requires $\bar{F} > 0.95$.
\end{theorem}

\begin{proof}
Fidelity at each time $t$ quantifies agreement between measured and theoretical trajectories. Perfect agreement gives $F(t) = 1$. Complete disagreement gives $F(t) = 0$.

Average over trajectory provides overall measure:
\begin{equation}
\bar{F} = \frac{1}{T}\int_0^T \left(\sum_i c_i^{\text{meas}}(t) c_i^{\text{theory}}(t)\right)^2 dt
\end{equation}

For $\bar{F} > 0.95$, the measured trajectory matches theory to within 5\% on average. This indicates successful ternary state reconstruction.

Experimental validation on CH$_4^+$ yields $\bar{F} = 0.983$, exceeding threshold.
\end{proof}

\subsection{Poincar\'e Recurrence and Trajectory Completion}

\begin{theorem}[Trajectory Completion]
\label{thm:trajectory_completion}
For bounded trajectory in $\Sspace$, Poincar\'e recurrence guarantees return to initial conditions. Molecular identification corresponds to finding trajectory satisfying recurrence conditions and matching measured spectral data.
\end{theorem}

\begin{proof}
By Theorem \ref{thm:trajectory_bounded}, molecular trajectory remains in bounded region $\Sspace = [0,1]^3$. By Poincar\'e recurrence theorem (Theorem \ref{thm:poincare}), almost every trajectory returns arbitrarily close to initial conditions infinitely often.

Define recurrence time $T_{\text{rec}}$ as time for trajectory to return within $\epsilon$ of initial point: $|\mathbf{s}(T_{\text{rec}}) - \mathbf{s}(0)| < \epsilon$. For molecular systems, $T_{\text{rec}}$ is determined by vibrational periods and electronic lifetimes.

Trajectory completion problem: Given measured spectral data $\{\nu_i, I_i\}$ (frequencies and intensities), find trajectory $\mathbf{s}(t)$ in $\Sspace$ that:
\begin{enumerate}
\item Satisfies rate equations (Theorem \ref{thm:ternary_evolution})
\item Matches measured frequencies: $\nu_i = \omega_i/(2\pi c)$ where $\omega_i$ are oscillation frequencies along trajectory
\item Satisfies recurrence: $\mathbf{s}(T_{\text{rec}}) \approx \mathbf{s}(0)$
\item Maintains boundedness: $\mathbf{s}(t) \in [0,1]^3$ for all $t$
\end{enumerate}

The trajectory is unique (up to phase) due to deterministic dynamics. Finding this trajectory constitutes complete molecular identification.
\end{proof}
